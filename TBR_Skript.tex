\documentclass[paper=a4, fontsize=11pt]{scrartcl}
\usepackage[utf8]{inputenc}
\usepackage{amsmath}
\usepackage{amsfonts}
\usepackage{amssymb}
\author{Kim Thuong Ngo}


\usepackage[T1]{fontenc}
\usepackage{fourier}

\usepackage{lipsum}

\usepackage{listings}
\usepackage{graphicx}
\usepackage{tabularx}

\usepackage{sectsty}
\allsectionsfont{\centering \normalfont\scshape}

\usepackage{fancyhdr}
\pagestyle{fancyplain}
\fancyhead{}
\fancyfoot[L]{}
\fancyfoot[C]{}
\fancyfoot[R]{\thepage}
\renewcommand{\headrulewidth}{0pt}
\renewcommand{\footrulewidth}{0pt}
\setlength{\headheight}{13.6pt}

\numberwithin{equation}{section}
\numberwithin{figure}{section}
\numberwithin{table}{section}

\setlength\parindent{0pt}

\newcommand{\horrule}[1]{\rule{\linewidth}{#1}}

\title{
\normalfont \normalsize
\textsc{Technik des betrieblichen Rechnungswesens} \\ [25pt]
\horrule{0.5pt} \\[0.4cm]
\huge Skript \\
\horrule{2pt} \\[0.5cm]
}

\author{Kim Thuong Ngo}

\date{\normalsize\today}

\begin{document}

\maketitle

\newpage

\tableofcontents

\newpage

%----------------------------------------------------------------------------------------

\newpage

\section{Grundlagen}

%-------------------------------------------------------------------

\subsection{Grundbegriffe}

%-------------------------------------------------------------------

\subsection{Rechnungslegungszwecke}

%-------------------------------------------------------------------

\subsection{Teilbereiche des Rechnungswesens}

%-------------------------------------------------------------------

\subsection{Gesetzliche Regelungen}

%-------------------------------------------------------------------

\subsection{Grundsätze ordnungsmäßiger Buchführung}

%-------------------------------------------------------------------

\subsection{Bestands- und Stromgrößen}

%----------------------------------------------------------------------------------------

\newpage

\section{Das System der doppelten Buchführung}

%-------------------------------------------------------------------

\subsection{Kaufmännische Buchführungssysteme}

%-------------------------------------------------------------------

\subsection{Inventur und Inventar}

%-------------------------------------------------------------------

\subsection{Bilanz}

%-------------------------------------------------------------------

\subsection{Technik erfolgsneutraler Buchungsvorgänge}

%-------------------------------------------------------------------

\subsection{Technick erfolgswirksamer Buchungsvorgänge}

\underline{erfolgswirksame Geschäftsvorfälle}

\begin{enumerate}
\item Verbuchung einer Zinsgutschrift (V)
BS: Bank an Zinsertrag
\item Auflösung der Rückstellung für Prozesskosten
BS: Rückstellungen an sonstigen betrieblichen Ertrag
\item Verbuchung einer Mietzahlung
BS: Mietprotokoll der Bank
\item Bildung einer Rückstellung für Prozesskosten
BS: Aufwand an Rückstellungen
\end{enumerate}

\underline{Mietaufwand}

\begin{tabular}{cc|cc}
Soll & & & Haben \\
\hline
1) & 16.000 & Saldo & 18.000 \\
\hline
 & 18.000 & & 18.000 \\
\end{tabular}

\underline{L2G Aufwand}

\begin{tabular}{cc|cc}
Soll & & & Haben \\
\hline
2) & 30.000 & 6) Saldo & 30.000 \\
\hline
& 30.000 & & 30.000 \\
\end{tabular}

\underline{GuV Konto}

\begin{tabular}{cc|cc}
Soll & & & Haben \\
\hline
5) & 18.000 & 7)  & 60-000\\
6) & 30.000 & 5) & 2.000 \\
9) & $\sum$ 14.000  & & \\
\hline
& 62.000 & & 62.000 \\
\end{tabular}

\underline{Bank}

\begin{tabular}{cc|cc}
Soll & & & Haben \\
\hline
AB & 50.000 & 1) & 18.000 \\
3) & 60.000 & 2) & 30.000 \\
4) & 2.000 & & \\
\hline
& 112.000 & & 112.000 \\
\end{tabular}

\underline{Umsatzerlöse}

\begin{tabular}{cc|cc}
Soll & & & Haben \\
\hline
1) Saldo & 60.000 & 3) & 60.000 \\
\hline
 & 60.000 & & 60.000 \\
\end{tabular}

\underline{Zinsertrag}

\begin{tabular}{cc|cc}
Soll & & & Haben \\
\hline
5) Saldo & 2.000 & 4) & 2.000 \\
\hline
& 2.000 & & 2.000 \\
\end{tabular}

\underline{Eigenkapital Konto}

\begin{tabular}{cc|cc}
Soll & & & Haben \\
\hline
EB & 24.000 & AB & 10.000 \\
& & 9) & 14.000 \\
\hline
&24.000 && 24.000 \\
\end{tabular}

%-------------------------------------------------------------------

\subsection{Private Buchungsvorgänge}

%-------------------------------------------------------------------

\subsection{Organisatorische Grundlagen}

%----------------------------------------------------------------------------------------

\newpage

\section{Laufende Geschäftsvorfälle}

%-------------------------------------------------------------------

\subsection{Warenverkehr, Materialverbrauch, Erzeugnisbestände}

%-------------------------------------------------------------------

\subsubsection{Buchung des Warenverkehrs}

%-------------------------------------------

\subsubsection{Einbeziehung der Umsatzsteuer}

%-------------------------------------------

\subsubsection{Anschaffungsnebenkosten}

%-------------------------------------------

\subsubsection{Retouren und Preisnachlässe}

%-------------------------------------------

\subsubsection{Eigenverbrauch}

%-------------------------------------------

\subsubsection{Anzahlungen}

%-------------------------------------------

\subsubsection{Verbrauch von Stoffen}

%-------------------------------------------

\subsubsection{Bestandsveränderungen von Erzeugnissen}

%-------------------------------------------------------------------

\subsection{Lohn und Gehalt}

%-------------------------------------------

\subsubsection{Institutionelle Grundlagen}

%-------------------------------------------

\subsubsection{Buchungstechnik}

%-------------------------------------------

\subsubsection{Geringfügiges Beschäftigungsverhältnis}

%-------------------------------------------

\subsubsection{Vorschüsse und Sachbezüge}

%-------------------------------------------

\subsubsection{Vermögenswirksame Leistungen}

%----------------------------------------------------------------------------------------

\newpage

\section{Vorbereitende Abschlussbuchungen}

%-------------------------------------------------------------------

\subsection{Anlagevermögen}

%-------------------------------------------

\subsubsection{Vorbemerkungen}

%-------------------------------------------

\subsubsection{Planmäßige Abschreibungen auf Sachanlagen}

%-------------------------------------------

\subsubsection{Außerplanmäßige Abschreibungen}

%-------------------------------------------

\subsubsection{Zuschreibungen}

%-------------------------------------------

\subsubsection{Veräußerung von Anlagevermögen}

%-------------------------------------------------------------------

\subsection{Umlaufvermögen}

%-------------------------------------------

\subsubsection{Handelswaren, Roh-, Hilfs- und Betriebsstoffe sowie Erzeugnisse}

%-------------------------------------------

\subsubsection{Forderungen}

%-------------------------------------------------------------------

\subsection{Zeitliche Abgrenzungen}

%-------------------------------------------

\subsubsection{Rechnungsabgrenzung}

%-------------------------------------------

\subsubsection{Darlehen und Disagio}

%-------------------------------------------

\subsubsection{Rückstellungen}

%-------------------------------------------

\subsubsection{Latente Steuern}

%----------------------------------------------------------------------------------------

\newpage

\section{Jahresabschluss}

%-------------------------------------------------------------------

\subsection{Abschlussbuchungen}

%-------------------------------------------------------------------

\subsection{Buchung des Erfolgs}

%----------------------------------------------------------------------------------------



\end{document}
