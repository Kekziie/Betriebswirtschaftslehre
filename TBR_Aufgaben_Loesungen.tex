\documentclass[paper=a4, fontsize=11pt]{scrartcl} 
\usepackage[utf8]{inputenc}
\usepackage{amsmath}
\usepackage{amsfonts}
\usepackage{amssymb}
\author{Kim Thuong Ngo}
\usepackage[T1]{fontenc} 
\usepackage{fourier} 
\usepackage{lipsum} 
\usepackage{listings}
\usepackage{graphicx}
\usepackage{tabularx}
\usepackage{sectsty}
\usepackage{xcolor}
\usepackage{wasysym}
\usepackage{stmaryrd}
\allsectionsfont{\centering \normalfont\scshape} 
\usepackage{fancyhdr} 
\usepackage{cancel}
\pagestyle{fancyplain} 
\fancyhead{}
\fancyfoot[L]{} 
\fancyfoot[C]{} 
\fancyfoot[R]{\thepage} 
\renewcommand{\headrulewidth}{0pt} 
\renewcommand{\footrulewidth}{0pt}
\setlength{\headheight}{13.6pt}

\numberwithin{equation}{section} 
\numberwithin{figure}{section} 
\numberwithin{table}{section}

\setlength\parindent{0pt}  

\newcommand{\horrule}[1]{\rule{\linewidth}{#1}} 

\title{	
\normalfont \normalsize 
\textsc{Technik des betrieblichen Rechnungswesen} \\ [25pt] 
\horrule{0.5pt} \\[0.4cm] 
\huge Aufgaben und Lösungen\\ 
\horrule{2pt} \\[0.5cm] 
}

\author{Kim Thuong Ngo} 
\date{\normalsize\today} 
\usepackage{color}


\begin{document}
\maketitle
\newpage
\tableofcontents
\newpage
%----------------------------------------------------------------------------------------
\section{Grundlagen}
%-------------------------------------------
\paragraph{Aufgabe 1} 
Nennen und erläutern Sie jeweils zwei rechtliche und ökonomische Zwecke der Rechnungslegung. \\

\underline{Rechtliche Zwecke:}
\begin{itemize}
\item Dokumentationsfunktion: \\
Breze
Die lückenlose Erfassung dient der Beweissicherung. Die Bücher können zur Schlichtung von Konflikten vor Gerichten dienen.

\item Anspruchsbemessungsfunktion:
  \begin{itemize}
  \item Ausschüttungsbemessungsfunktion \\
Mit Hilfe der Rechnungslegung wird die Größe "Gewinn" (bzw. "Verlust") ermittelt. Auf  der Basis des Gewinns wird bestimmt, welcher Teil des Vermögens den Eigenen zur Ausschüttung zur Verfügung steht.
  \item Steuerbemessung: \\
Die von den Unternehmen zu zahlenden Steuern bemessen sich in Deutschland nach der Größe "Gewinn" (bzw. "Verlust") bei Beachtung der steuerrechtlichen Vorgaben.
  \item Gläubigerschutz: \\
Je weniger Vermögen an die Eigner ausgeschüttet wird, desto mehr verbleibt im Unternehmen und steht für die Befriedigung der Ansprüche der Gläubiger zur Verfügung. Das nicht ausgeschüttete Vermögen stellt eine Art "Polster" für Krisenzeiten dar. Je vorsichtiger und konservativer die Größe "Gewinn" (bzw. "Verlust") bestimmt wird, desto geringer die Ausschüttung an die Eigner.
\end{itemize}
\end{itemize}

\underline{Ökonomische Zwecke:}
\begin{itemize}
\item Informationsfunktion: \\
Die Offenlegung des Jahresabschlusses verringert die Informationssymmetrie zwischen der (internen) Unternehmungsleitung und der (externen) Eigen- und Fremdkapitalgebern. Der Jahresabschluss fasst die Informationen der Rechnungslegung aus den vergangenen Geschäftsjahr zusammen.

\item Anreizfunktion: \\
Die Rechnungslegung liefert Informationen, an die die Vergütung der Unternehmungsleitung anknüpfen kann. Beispielsweise hat ein Manager einen Anreiz den Gewinn zu steigern, wenn sich seine Vergütung am Gewinn orientiert.
\end{itemize}

%-------------------------------------------
\paragraph{Aufgabe 2}
Erläutern Sie inwiefern sich Rechnungslegungszwecke und die Interesse der Rechnungslegungsadressaten widersprechen können. \\

Konfliktäre Interessen der Adressaten und Zwecke der Rechnunglegung: \\
\begin{itemize}
\item z.B. Gläubiger, \\
an einem nicht zu hohen Schuldenstand des Unternehmens interessiert \\
$\rightarrow$ vorsichtige Rechnungslegung führt zur geringeren Ausschüttungen

\item z.B. Eigenkapitalgeber,
  \begin{itemize}
  \item fordern im Gegensatz zu Gläubigern Ausschüttungen \\
$\rightarrow$ weniger vorsichtige Rechnungslegung
  \item fordern Informationen für Kapitalanlagenentscheidungen, die nciht durch eine (übermäßige) vorsichtige Rechnungslegung verzerrt sein sollten
  \item fordern für Kapitalanlagenentscheidungen eher zukunftsoriertierte Informationen
\end{itemize}

\item z.B Fiskus/ Gerichte: \\
Eher an justiziablen, vergangenheitsorientierten Größen interessiert \\
$\rightarrow$ Rechnungslegung mit eingeschränkten Ermessungsspielraum der Unternehmensführung \\
$\rightarrow$ problematisch:  benötigte Informationen unterschiedlicher Natur \\
  \begin{itemize}
  \item vergangenheitsorientiert (Dokumentation) \\
zukunftsorientiert (Information)
  \item für interne Zwecke (retrospektiver Soll-Ist-Vergleich) \\
für externe Zwecke (Information für Externe, z.b. Steuererhebung)
\end{itemize}
\end{itemize}

%-------------------------------------------
\paragraph{Aufgabe 3}
Erläutern Sie die (externe) Informationsfunktion der Rechnungslegung. \\

\underline{Ausgangspunkt:}

Informationsassymetrie zwischen Managern und Eignern börsenorientierter Publikumsgesellschaften aufgrund von der Trennung von Eigentum und Geschäftsführung.
Eigenkapitalgeber stellen externe Adressaten der Recnungslegung dar. \\

\underline{Informationsvorprung des Managers:}

\begin{itemize}
  \item künftige Erfolge der Projekte
  \item das eigene Verhalten
  \item Erfolge bereits durchgeführter Projekte
\end{itemize}

Informationen im Jahresabschluss dienen dem Abbau der Informationassymetrie durch:

\begin{itemize}
  \item Bereitstellung der Informationen für Kapitalanlageentscheidungen
  \item Rechenschaftslegung
\end{itemize}

Ziel: Gewährleistung der Funktionsfähigkeit des Kapitalmarktes:

\begin{itemize}
  \item geringere Informationsassymetrie reduziert die Notwendigkeit der Kapitalmarktteilnehmer
  sich gegen Übervortelung durch das Management zu schützen, was Kosten verursacht
  \item Kosten können prohibitiv hoch sein, sodass der Kapitalmarkt zusammenbricht
\end{itemize}

%-------------------------------------------
\paragraph{Aufgabe 4}
Paragraf 58 des Aktiensgesetzes regelt die Verwendung des Jahresüberschusses in der Aktiengesellschaft. Unter anderem sind folgende Regelungen enthalten:
\begin{itemize}
\item Absatz 2: Stellen Vorstand und Aufsichtsrat den Jahresabschluss fest, so können sie einen Teil des Jahresüberschusses, höchstens jedoch die Hälfte, in andere Gewinnrücklagen einstellen.
\item Absatz 3: Die Hauptversammlung kann im Beschluss über die Verwendung des Bilanzgewinns weitere Beträge in Gewinnrücklagen einstellen oder als Gewinn vortragen.
\item Absatz 4: Die Aktionäre haben Ansprüche auf den Bilanzgewinn, soweit er nicht nach Gesetz oder Satzung durch Hauptversammlungsbeschluss nach Absatz3 oder als zusätzlicher Aufwand auf Grund des Gewinnverwendungsbeschlusses von der Verteilung unter die Aktionäre ausgeschlossen ist.
\end{itemize}
Erläutern Sie, wie der Rechnungslegungszweck der Anspruchsbemessung durch Paragraf 58 AktG konkretisiert wird. \\

%-------------------------------------------
\paragraph{Aufgabe 5}
Nennen und erläutern Sie die zwei gundlegenden Teilbereiche des Rechnungswesens. \\

\underline{Internes Rechnungswesens}
\begin{itemize}
  \item Allgemein :\\
  Informationsbereitstellung zur Entscheidungsfindung für Unternehmensführung und
  Controlling (Selbstinformation)
  \item Teilbereiche:
  \begin{itemize}
    \item Kostenrechnung:
    \begin{itemize}
      \item Ziel: \\
      interne Rechenschaftslegung einzelner Abteilungen und Ermittlungen von Wertansätzen
      (gilt auch für externes Rechnungswesen)
      \item konkret:
      \begin{itemize}
        \item Kostenarten (welche Kosten?)
        \item Kostenstellenrechnung (wo fallen Kosten an?)
        \item Kostenträgerrechnung (wofür?/ wie hoch?)
      \end{itemize}
    \end{itemize}
    \item Statistiken/ Vergleichsrechnungen: \\
    Sammlung und Aufbereitung interner und externer Daten für
    Unternehmensleitung (Umsatz, Nachfrage, ...)
    \item Planungsrechnung: \\
    Verwendung interner und externer Daten zur Prognose der künftigen Entwicklungen \\
    $\rightarrow$ Wirtschaftlichkeitsrechnung, Produktions- und Investitionsentscheidungen
  \end{itemize}
\end{itemize}

\underline{externes Rechnungswesen:}

\begin{itemize}
  \item Informationsbereitstellung für externe Adressaten (z.B. Eigenkapitalgeber)
  \item eine alternative Beziehung: "Finanzbuchahltung" (FB)
  \item FB besteht aus:
  \begin{itemize}
    \item Buchführung
    \item Inventar
    \item Jahresabschluss: Bilanz (= Vermögensübersicht), Gewinn- und Verlustrechnung
    \item Sonderbilanz
  \end{itemize}
\end{itemize}

Starke gesetzliche Reglementierung durch Vorschriften!!

%-------------------------------------------
\paragraph{Aufgabe 6}
Erläutern Sie, was unter Finanzbuchhaltung zu verstehen ist. \\
\begin{itemize}
\item Finanzbuchhaltung: Synonym für externes Rechnungswesen 
\item gesetzlich stark reguliert (durch Handelsgesetzbuch HGB)
\item Bestandteil
 \begin{itemize}
 \item Buchführung: dort werden alle Geschäftsvorfälle systematisch, lückenlos, zahlen- und wertmäßig erfasst
 \item Inventar: Ergebnis einer detaillierten Bestandsaufnahme, der Inventur
 \item Jahresabschluss: umfasst die Bilanz, sowie Gewinn- und Verlustrechnung; Kapitalgesellschaften haben zusätzlich einen Anhang zu erstellen, der die Bilanz und GuV erklärt
 \item Sonderbilanzen: z.B. Sanierungs-, Gründungs-,Umwandlungs-,Liguiditationsbilanz
 \end{itemize}
\end{itemize}
%-------------------------------------------
\paragraph{Aufgabe 7}
Erläutern Sie kurz, was unter induktiver und deduktiver Ableitung von Grundsätzen ordnungsmäßiger Buchführung (GoB) zu verstehen ist. \\

GoB: allgemeine Regeln zur Buchführung in Deutschland \\
$\rightarrow$ geschriebenr und ungeschriebenr Teil (HGB) \\

\underline{Induktive Ableitung} 
\begin{itemize}
\item Grundsätze ordnungsmäßiger Buchführung werden basierend auf der Verkehrsauffassung oder dem Handelsgebrauch festgelegt
\item das tatsächliche Verhalten der Kaufleute im Wirtschaftsgeschehen/-verkehr und das, was ordentlich und ehrbare Kaufleute für richtig halten, bildet die Grundlage für die Ermittlung der GoB
\end{itemize}

\underline{Deduktive Ableitung} 
\begin{itemize}
\item ausgehend von den Rechnungslegungszwecken, wie sie sich aus dem Gesetz, der Rechtssprechung und der betriebswirtschaftlichen Fachliteratur ergeben, werden die GoB ermittelt
\item ordnungsgemäße Buchführungsgrundsätze sind demnach so auszugestalten, dass die Errechnung der Rechnungslegungszwecke gewährleisten 
\end{itemize}

$\rightarrow$ die herrschende Meinung befürwortet die Ableitung der GoB nach der deduktiven Methode

%-------------------------------------------
\paragraph{Aufgabe 8}
Geben Sie zu jedem Geschäftsvorfall an, welche Stromgrößen betroffen sind und welche Bestandsgrößen sich in welche Richtung verändern: \\
\begin{enumerate}
\item Barverkauf von Handelswaren zum Wareneinkaufspreis \\
Einzahlung, Entnahme (ZM $\uparrow$, GV $\uparrow$, RV $\rightarrow$)
\item Erhalt von Mietzahlungen \\
Einzahlung, Einnahme (ZM $\uparrow$, GV $\uparrow$, RV $\uparrow$)
\item Bareinkauf von Rohstoffen, die gelagert werden \\
Auszahlung, Ausgabe (ZM $\downarrow$, GV $\downarrow$, RV $\rightarrow$)
\item Kauf eines LKW auf Ziel \\
auf Ziel: man zahlt später $\rightarrow$ nimmt Verbindlichkeit auf \\
Ausgabe, keine Auszahlung (ZM $\downarrow$, GV $\downarrow$, RV $\rightarrow$)
\item Wertaufholung einer zuvor außerplanmäßig abgeschriebenen Maschine \\
Abschreibungen bilden den Wertverzehr eines Vermögensgegenstands 
\begin{itemize}
\item[i)] planmäßig: regelmäßiger Verbrauch/ Abnutzung
\item[ii)] außerplanmäßig: Unfälle/ Verschleiß \\
Wertaufholung: für außerplanmäßige Abschreibung fällt weg
\end{itemize}
Ertrag
\item Kauf von Treibstoff auf Ziel, der unverzüglich verbraucht wird \\
Ausgabe, Aufwand (RV $\downarrow$)
\item Barverkauf eines betrieblich genutzten PKW zum Buchwert \\
Einzahlung, Einnahme (GV $\uparrow$, ZM $\uparrow$)
\item Barkauf einer Maschine \\
Auszahlung, Ausgabe (GV $\downarrow$, ZM $\downarrow$)
\item Tilgung eines Kredits (ohne Zinszahlungen) \\
Auszahlung, Ausgabe (GV $\downarrow$, ZM $\downarrow$)
\item Verkauf einer Maschine mit einem Buchwert von null auf Ziel \\
Einnahme, Ertrag (GV $\uparrow$, ZM $\uparrow$)
\item Kauf von Rohstoffen, die gelagert werden, auf Ziel \\
Ausgabe (GV $\downarrow$)
\item Auszahlung von Löhnen \\
Auszahlung, Ausgabe, Aufwand (GV $\downarrow$,RV $\downarrow$, ZM $\downarrow$)
\item Aufnahme eines Kredits \\
Einzahlung (BM $\uparrow$)
\item Verkauf von Handelswaren zum Wareneinkaufspreis auf Ziel \\
Einnahme (GV $\uparrow$)
\item Verkauf von Büromöbeln zum Buchwert auf Ziel \\
Einnahme (GV $\uparrow$)
\item Planmäßige Abschreibung einer Maschine \\
Aufwand (RV $\downarrow$)
\end{enumerate}

%-------------------------------------------
\paragraph{Aufgabe 9}
Im Folgenden sind Kombinationen von Bestandsveränderungen und/oder Stromgrößen angegeben, die zu Bestandsveränderungen führen bzw. nicht führen. Geben Sie für jeden Fall ein Beispiel für einen passenden Geschäftsvorfall an: \\
\begin{enumerate}
\item Zahlungsmittel sinken, keine Ausgabe \\
Barbegleichung einer Verbindlichkeit
\item Auszahlung, Ausgabe, Reinvermögen unverändert \\
Barverkauf einer Lagerung von Rohstoffen
\item Einzahlung, Einnahme, Reinvermögen unverändert \\
Barkauf von Fertigungserzeugnissen
\item Geldvermögen steigt, Reinvermögen unverändert, keine Einzahlung \\
Verkauf von Fertigungserzeugnissen zum Buchwert auf Ziel
\item Ausgabe, keine Auszahlung, Reinvermögen unverändert \\
Kauf eines betrieblich genutzten PKWs auf Ziel
\item Reinvermögen unverändert, Einzahlung, Geldvermögen steigt \\
Barverkauf von Handelswaren zum Wareneinkaufspreis
\item Geldvermögen und Reinvermögen sinken, keine Auszahlung \\
Kauf von Treibstoff auf Ziel, der sofort verkauft wird
\item Geldvermögen unverändert, Aufwand \\
Abschreibung einer Maschine
\item Geldvermögen und Reinvermögen sinken, keine Auszahlung \\
Kauf von Bürogenständen auf Ziel
\item Reinvermögen und Geldvermögen steigen, Einzahlung \\
Erhalt von Mietzahlungen
\item Zahlungsmittel und Geldvermögen sinken, kein Aufwand \\
Barkauf einer PKWs für betriebliche Zwecke
\item Reinvermögen steigt, keine Einzahlung, Geldvermögen steigt \\
Verkauf einer Maschine mit Buchwert oder auf Ziel
\item Ertrag, keine Einnahme \\
Werteaufholung einer zuvor außerplanmäßig abgeschriebenen Maschine 
\item Einzahlung, keine Einnahme, Reinvermögen unverändert \\
Kunde begleicht Forderung
\item Einnahme, Reinvermögen und Zahlungsmittel unverändert \\
Verkauf von Betriebs- und Geschäftsausstattung zum Buchwert auf Ziel
\item Ausgabe, kein Aufwand, keine Auszahlung \\
Kauf eines Computers auf Ziel
\end{enumerate}

%----------------------------------------------------------------------------------------
\section{Das System der doppelten Buchführung}
%-------------------------------------------
\paragraph{Aufgabe 1}
Erläutern Sie das System der Doppik. \\

Doppelte Buchführung 
\begin{itemize}
\item Erfassung aller das Vermögen (auch das Eigenkapitals) beführende Geschäftsvorfälle in zeitlicher (Grundbuch=Journals) und sachlicher Ordnung (Hauptbuch)
\item Buchungen finden aus Kontrollgründen stets im Soll und im Haben statt (Soll an Haben!)
\item es existieren sowohl Bestands- als auch Erfolgskonten
\item Entwicklung einer Vermögensübersicht (Bilanz) möglich
\item Entwicklung einer Erfolgsübersicht wird ermöglicht sowohl durch: 
\item Eigenkapitalsvergleich (Erfolg-Endbestand EK-Anfangsbestand EK-Einlagen+Entnahme)
\item als auch über die Gewinn- und Verlustrechnung
\item Möglichkeit der Kontrolle, weil beide Arten der Erfolgsermittlung dasselbe Ergebnis liefern müssen
\end{itemize}
%-------------------------------------------
\paragraph{Aufgabe 2}
Ihnen liegt folgende Übersicht über den Anfangsbestand sowie Zu- und Abgänge an Tennisschlägern eines Sportgeschäfts vor. Der Bilanzstichtag entspricht dem 31.12. Der Wert bzw. der Verkaufs- oder Einkaufspreis eines Tennisschlägers beträgt konstant $250 €$. Ein Kassenbestand, Bankguthaben, Forderungen oder Verbindlichkeiten bleiben unberücksichtigt. \\
\includegraphics[width=10cm,height=11cm]{Tennis.png} \\
\begin{itemize}
\item[a)] Erstellen Sie das Inventar zum Bilanzstichtag im Jahr J1. Führen Sie eine klassische Stichtagsinventur durch. \\

Inventar zum 31.12.J1 \\
276 Tennisschläger zu je 250 € \\
$\rightarrow$ 69.000 € = 276 * 250 € 

\item[b)] Führen Sie eine permanente Inventur auf das Bilanzstichtag im Jahr J1 durch. Der Inventurtag fällt auf den 30.06.J1. \\

Inventurtag 30.06.J1 \\
Inventarstichtag 31.12.J1 \\
Bestand zum 30.06.J1: 32 \\
Vorrechnung: 32+244=276 \\
Inventar zum 31.12.J1: 276 Tennisschläger zu je 250 € = 69.000 € 

\item[c)] Ermitteln Sie den Bilanzwert der Tennisschläger zum Bilanzstichtag im Jahr J1. Der Stichtag für das besondere Inventar, der mit dem Inventurtag zusammenfällt, ist der 28.02.J2. \\

aus Teilaufgabe a) \\
\begin{tabular}{|c|c|c|}
\hline
 & Bestand zum 31.12.J1 & 276 Tennisschläger \\\hline
 Abgang & Januar J2 & -1 Tennisschläger \\
 Zugang & 31.01.J2 & +30 Tennisschläger \\
 Abgang & Februar J2 & -5 Tennisschläger \\\hline
 & Endbestand zum 28.02.J2 & 300 Tennisschläger * 250 € = 75.000 €
\end{tabular}
Besonderes Inventar zum 28.02.J2 \\
300 Tennisschläger zu je 250 € $\rightarrow$ 75.000 € \\

Besonderheit der (vor- oder) nachgelegten Stichtagesinventur: Nur Erfassung der Veränderung zwischen 31.12.J1 und dem 28.02.J2 (Rückrechnung) \\

\begin{tabular}{|c|}
\hline
Inventurtag: \\
Stichtag für die besondere Inventur 28.02.J2 \\
Bilanzstichtag: 31.12.J1 (nachgelagerte Stichtagsinventur) \\
\hline
\end{tabular}

Bilanzwert am 31.12.J1 \\
75.000 € besonderem Inventar \\
75.000 € - 24 Stück * 250 € = \underline{69.000 €} \\

\item[d)] Ermitteln Sie den Bilanzwert der Tennisschläger zum Bilanzstichtag im Jahr J1. Der Stichtag für das besondere Inventar ist der 31.10.J1. Der 31.08.J1 ist der Inventurtag (permanente Inventur auf den Stichtag des besonderen Inventars)

aus Teilaufgabe b) \\
\begin{tabular}{|c|c|c|}
\hline
 & Bestand zum 30.06.J1 & 32 Tennisschläger \\\hline
 Zugang & Juli J1 & 210 Tennisschläger \\
 Abgang & 11.- 31.07.J1 & -30 Tennisschläger \\
 Abgang & August J1 & -20 Tennisschläger \\\hline
 & Endbestand zum 31.08.J1 & 192 Tennisschläger \\\hline
\end{tabular}

\underline{Zu-/ Abgänge zwischen 31.09.J1 und dem 31.10.J1} \\
\begin{tabular}{cc|c}
Abgang & September J1 & -23 Tennisschläger \\
Abgang & Oktober J1 & -10 Tennisschläger \\\hline
 & Differenz & -33 Tennisschläger \\
\end{tabular}

\underline{Endbestand zum 31.10.J1} \\
192-33=159 \\
Besonderes Inventar zum 31.10.J1 \\
159 Tennisschläger * 250 € = 39.750 € \\

\underline{Veränderung zwischen 01.11.J1 und 31.12.J1 (Vorrechnung)} \\
\begin{tabular}{cc|c}
Abgang & November J1 & -8 Tennisschläger \\
Zugang & zum 31.11.J1 & 150 Tennisschläger \\
Abgang & Dezember J1 & -25 Tennisschläger \\\hline
 & Differenz & 117 Tennisschläger 
\end{tabular}

\begin{tabular}{|c|}
\hline
Inventurtag: 31.08.J1 \\
Stichtag für das besondere Inventar: 31.10.J1 \\
Bilanzstichtag: 31.12.J1 \\
\hline
\end{tabular}

\underline{Bilanzwert zum 31.12.J1} \\
39.750 € besonderes Inventar \\
39.750 € + 117 Stück * 250 € = 69.000 €
\end{itemize}
%-------------------------------------------
\paragraph{Aufgabe 3}
Erläutern Sie die Begriffe Aktivtausch, Passivtausch, Bilanzverlängerung und Bilanzverkürzung und geben Sie jeweils zwei Beispiele in Form eines erfolgsneutralen Buchungssatzes an. \\

\paragraph{Aktivtausch}
\begin{itemize}
\item Änderung der Struktur der Aktivseite
\item Bilanzsumme bleibt unverändert
\item Passivseite unverändert
\end{itemize}
Beispiel: Wareneinkauf bar oder Bank
\paragraph{Passivtausch}
\begin{itemize}
\item Änderung der Struktur der Passivseite
\item Bilanzsumm unverändert
\item Aktivseite unverändert
\end{itemize}
Bsp: Kredittilgung bei Bank A durch zusätzliche Kreditaufnahme bei Bank B
\paragraph{Bilanzverlängerung(Aktive/Passivmehrung)}
\begin{itemize}
\item Bestandszunahme auf Aktiv- und Passivseite in gleicher Höhe
\item Bilanzsumme steigt
\end{itemize}
Bsp: Wareneinkauf auf Ziel
\paragraph{Bilanzverkürzung(Aktive/Passivminderung)}
\begin{itemize}
\item Bestandsabnahme auf der Aktiv- und Passivseite in gleicher Höhe
\item Bilanzsumme sinkt
\end{itemize}
Bsp: Kredittilgung durch Barzahlung

\paragraph{Beispiele}
\begin{enumerate}
\item a) Bank an Kasse (Bargeldeinzahlung auf Bank \\
b) Kasse an Forderungen aus LuL 
\item a) Verbindl. aus LuL an Verbindl. ggü Lieferung\\
b) Verbindl. ggü KI x an Verbindl. ggü KI y (Umschuldung)
\item a) Kasse an Verbindl. ggü KI \\
b) Technische Anlagen und Maschinen an VLL (Kauf einer Maschine auf Ziel)
\item a) Verbindl. ggü KI an Bank (Kredittilgung durch Überweisung) \\
b) Verbindl. aus LuL an Kasse (Begleichung einer Verbindl. aus LuL durch Barzahlung)
\end{enumerate}
%-------------------------------------------
\paragraph{Aufgabe 4}
Ihnen liegen die unten aufgeführte Eröffnungsbilanz und die (erfolgsneutralen) Geschäftsvorfälle des Geschäftsjahres J1 vor. Führen Sie (1) die Eröffnungsbuchungen, (2) die laufenden Buchungen und (3) die Abschlussbuchungen durch. Bilden Sie die Geschäftsvorfälle mit Buchungssätzen ab und geben Sie die berührten Konten in T-Konten-Form an. Die Umsatzsteuer ist zu vernachlässigen. \\
\includegraphics[width=12cm,height=4cm]{Bilanz.png} \\
Geschäftsvorfälle im Jahr J1: \\
\begin{enumerate}
\item Ein betrieblich genutzter PKW wird zum Preis von 10.000 € gekauft. Der Kaufpreis
wird in Höhe von 7.000 € per Banküberweisung beglichen. Der Rest wird bar bezahlt.
\item Ein Kunde begleicht eine Forderung. Er überweist 2.000 € auf das Bankkonto.
\item Es wird ein Darlehen in Höhe von 30.000 € aufgenommen. Der Darlehensgeber überlässt dem Unternehmen eine Forderung gegenüber Dritten in Höhe von 5.000 €, Wertpapiere im Wert von 10.000 € sowie Barmittel in Höhe von 7.000 €. Der Rest wird auf
das Bankkonto überwiesen.
\item Die Wertpapiere werden zu einem Preis von 10.000 € (= Buchwert der Wertpapiere)
verkauft. Bankspesen fallen nicht an. Der Verkaufserlös wird dem Bankkonto gutgeschrieben.
\item Durch Bezahlung einer Rate in Höhe von 5.000 € per Banküberweisung wird ein Teil
des Darlehens getilgt (Zinsen sind zu vernachlässigen).
Büromöbel im Wert von 1.500 € werden gekauft und bar bezahlt.
\item Ein Darlehen wird aufgenommen um Verbindlichkeiten aus L\&L in Höhe von 12.000
€ zu begleichen.
\item Waren im Wert von 20.000 € werden zum Buchwert verkauft. Der Kunde überweist
den Rechnungsbetrag auf das Bankkonto.
\item Es werden 300 € aus der Kasse auf das Bankkonto eingezahlt.
\item Das Unternehmen kauft Waren im Wert von 4.000 € ein. Der Lieferant erhält 1.000 €
in bar. Der Rest wird zur Hälfte kreditiert und zur Hälfte per Banküberweisung beglichen.
\item Forderungen in Höhe von 2.500 € werden vom Lieferanten bar beglichen.
\item Das Unternehmen überweist 7.500 € an seine Lieferanten und reduziert dadurch seine Lieferantenverbindlichkeiten.
\item Ein Gebäude wird zum Buchwert von 45.000 € verkauft. Mit dem Verkauf geht eine
\item Darlehensschuld in Höhe von 10.000 € auf den Käufer über. Das Unternehmen erhält
den restlichen Kaufpreis per Banküberweisung.
\end{enumerate}

\paragraph{Eröffnungsbuchung}
Konteneröffnung \\
\begin{tabular}{cc|cc}
S & & & H \\\hline
EK & 160.000 & Grund+Bauten & 100.000 \\
Verbindl. ggü KI & 55.000 & Betriebs- und GA & 5.000 \\
VLL & 30.000 & Waren & 85.000 \\  
  & & FLL & 6.000 \\
  & & Kasse & 4.000 \\  
  & & Bank & 45.000 \\\hline
  & 245.000 & & 245.000 \\\hline
\end{tabular} \\

Grund+Bauten an EBK 100.000 \\
BGA an EBK 5.000 \\
Waren an EBK 85.000 \\
FLL an EBK 6.000 \\
Kasse an EBK 4.000 \\
Bank an EBK 45.000 \\
EBK an EK 160.000 \\
EBK an VKI 55.000 \\
VLL an VLL 30.000 \\

\textbf{Grund+Bauten}
\begin{tabular}{cc|cc}
S & & & H \\\hline
AB & 100.000 & & \\
  & & & 45.000 \\\hline
  & 100.000 & & 45.000 \\\hline
\end{tabular} \\
\textbf{EK-Konto}
\begin{tabular}{cc|cc}
S & & & H \\\hline
  & & AB & 160.000 \\\hline
  & & & 160.000 \\\hline
\end{tabular} \\
\textbf{BGA}
\begin{tabular}{cc|cc}
S & & & H \\\hline
AB & 5.000 & & \\
6) & 1.500 & & \\\hline
  & 6.500 & & \\\hline
\end{tabular} \\
\textbf{Verb. ggü KI}
\begin{tabular}{cc|cc}
S & & & H \\\hline
 & & AB & 55.000 \\
12) & 7.500 & 3) &30.000 \\
5) & 5.000 & 7) & 12.000 \\\hline
  &12.500 & & 97.000 \\\hline
\end{tabular} \\
\textbf{Waren}
\begin{tabular}{cc|cc}
S & & & H \\\hline
AB & 85.000 & & \\
10) & 4.000 & 8) & 20.000 \\\hline
  & 89.000 & & 20.000 \\\hline
\end{tabular} \\
\textbf{Verb. aus LuL}
\begin{tabular}{cc|cc}
S & & & H \\\hline
7) & 12.000 &  AB & 30.000 \\
12) & 7.500 & 10) & 1500 \\\hline
  & 19500 & & 31.500 \\\hline
\end{tabular} \\
\textbf{Forderung aus LuL}
\begin{tabular}{cc|cc}
S & & & H \\\hline
AB & 6.000 & & \\
3) & 5.000 & 2) & 2.000 \\
13) & 10.000 & 11) & 2.500 \\\hline
  & 21.000 & & 4.500 \\\hline
\end{tabular} \\
\textbf{Kasse}
\begin{tabular}{cc|cc}
S & & & H \\\hline
  & & 1) & 3.000 \\
  & & 6) & 1.500 \\
  & & 9) & 300 \\
  & & 10) & 1.000 \\\hline
\end{tabular} \\
\textbf{Fuhrpark}
\begin{tabular}{cc|cc}
S & & & H \\\hline
1) & 10.000 & & \\\hline
  & 10.000 & & \\\hline
\end{tabular} \\
\textbf{Wertpapiere des UV}
\begin{tabular}{cc|cc}
S & & & H \\\hline
3) & 10.000 & 4) & 10.000 \\\hline
  & 10.000 & & 10.000 \\\hline
\end{tabular} \\
\textbf{Bank}
\begin{tabular}{cc|cc}
S & & & H \\\hline
AB & 45.000 & 1) & 7.000 \\
2) & 2.000 & 5) & 5.000 \\
3) & 8.000 & 10) & 2.000 \\
4) & 10.000 & 12) & 7.500 \\  
8) & 20.000 & & \\  
9)  300& & & \\  
13) & 35.000 & & \\\hline
  & 120.300 & & 21.000 \\\hline
\end{tabular} \\

\paragraph{Buchungssätze zu 2.4}
\begin{enumerate}
\item Fuhrpark 10.000 an Bank 7.000 \\
Kasse 3.000
\item Bank 2.000 an Forderung LuL
\item Forderungen aus LuL 5.000 \\
Wertpapiere 10.000 an Verbindlichkeiten ggü KI 30.000 \\
Kasse 7.000 \\
Bank 8.000
\item Bank 10.000 an Wertpapiere
\item Verbindlichkeiten ggü KI 5.000 an Bank 
\item BGAA 1.500 an Kasse
\item VLL an VKI 12.000
\item Bank an Waren 20.000
\item Bank an Kasse 300
\item Waren an Kasse \\
Bank 1500 \\
VLL 1500
\item Kasse an FLL 2500
\item Verbindlichkeiten ggü LuL an Bank 7500
\item Bank 35.000 an Grund+Bauten
\end{enumerate}

\paragraph{Schlussbilanzkonto}
\begin{tabular}{cc|cc}
S & & & H \\\hline
Grund+Bauten & 55.000 & EKK & 160.000 \\
BGA & 65.000 & Verb. ggü KI & 82.000 \\
Waren & 69.000 & Verb. ggü LuL & 12.000 \\
FLL & 6.500 & & \\
Kasse & 7.700 & & \\ 
Fuhrpark & 10.000 & & \\         
Bank & 99.300 & & \\\hline
 & 254.000 & & 254.000 \\\hline
\end{tabular}

\paragraph{Kontenabschluss}
\begin{enumerate}
\item SBK an Grund+Bauten 55.000
\item SBK an Betriebs+GA 6.500
\item SBK an Waren 69.000
\item SBK an FLL 6.500
\item SBK an Kasse 7.700
\item SBK an Fuhrpark 10.000
\item SBK an Bank 99.300
\item EKK an SBK 160.000
\item Verb. ggü KI an SBK 82.000
\item Verb. ggü LuL an SBK 12.000
\end{enumerate}
%-------------------------------------------
\paragraph{Aufgabe 5}
Bilden Sie die Buchungssätze für die folgenden Geschäftsvorfälle. Geben Sie jeweils an, welcher der vier Grundfälle erfolgsneutraler Geschäftsvorfälle vorliegt. Steuern sind zu vernachlässigen. \\
\begin{enumerate}
\item Barabhebung vom Bankkonto in Höhe von 1.000 €.
\item Bareinlage des Inhabers in Höhe von 25.000 €.
\item Die Hausbank sagt uns einen Kredit in Höhe von 12.000 € zu.
\item Der Kreditvertrag wird unterzeichnet und die 12.000 € auf das Firmenkonto überwiesen.
\item Zielkauf von Ware im Wert von 750 €.
\item Rückzahlung eines Darlehens in Höhe von 75.000 €.
\item Banküberweisung an einen Lieferanten in Höhe von 1.700 €.
\item Ein Darlehen wird aufgenommen um Verbindlichkeiten aus L\&L in Höhe von 4.000 € zu
begleichen.
\item Es werden Wertpapiere des UV im Wert von 2.750 € gegen Wertpapiere des AV eingetauscht.
\item Ein betrieblich genutzter PKW wird zum Buchwert von 2.000 € auf Ziel verkauft.
\item Der Käufer des PKW begleicht die Rechnung per Banküberweisung.
\item Eine Kundenforderung im Wert von 7.000 € wird an einen Lieferanten abgetreten.
\item Es wird ein betrieblich genutzter PKW zum Preis von 25.000 € gekauft. Ein älteres Modell wird dafür zum Buchwert in Zahlung gegeben. Die restlichen 17.500 € werden per
Banküberweisung bezahlt.
\item Mit einem Zulieferer wird ein langfristiger Liefervertrag mit einem Volumen von 50.000€ abgeschlossen.
\item Eine neue (leere) Lagerhalle im Wert von 100.000 € wird erworben. Die darauf lastende Hypothek in Höhe von 20.000 € wird übernommen, der Rest wird zu einem Viertel kreditiert und zu drei Vierteln per Banküberweisung bezahlt
\end{enumerate}

\begin{enumerate}
\item Kasse an Bank 1000 $\rightarrow$ AT
\item Kassen an Privatkonto 25.000 $\rightarrow$ AP+
\item keine Buchung
\item Bank an Verbindlichkeiten ggü KI 12.000 $\rightarrow$ AP+
\item Waren an Verbindlichkeiten aus LuL 750 $\rightarrow$ AP+
\item Verb. ggü KI an Bank 75.000 $\rightarrow$ AP-
\item Verb. aus LuL an Bank 1.700 $\rightarrow$ AP-
\item Verb. aus LuL an Verb. ggü Lieferanten 4.000  $\rightarrow$ PT
\item Wertpapiere des AV an Wertpapiere des UV 2750 $\rightarrow$ AT
\item Forderung aus LuL an Fuhrpark 2000 $\rightarrow$ AT
\item Bank an Forderung aus LuL 2000 $\rightarrow$ AT
\item Verb. aus LuL an Ford. aus LuL 7000 $\rightarrow$ AP-
\item Fuhrpark 25.000 an Fuhrpark 7500 $\rightarrow$ AT \\
Bank 17.500 $\rightarrow$ AT
\item keine Buchung
\item Grund+Bauten 100.000 an Verb. ggü KI (Hypothek) 20.000 $\rightarrow$ AP+ \\
Verb. ggü KI 20.000 $\rightarrow$ AP+ \\
Bank 60.000 $\rightarrow$ AT
\end{enumerate}
%-------------------------------------------
\paragraph{Aufgabe 6}
Geben Sie jeweils an, welcher Geschäftsvorfall zu den folgenden Buchungssätzen führt:
 \\
\begin{enumerate}
\item Kasse an Waren \\
Barverkauf von Waren zum Buchwert (zum selber Wert verkauft, wie eingekauft)
\item Bank an Wertpapiere des UV \\
Verkauf von Wertpapieren des UV zum Buchwert. Der Verkaufserlös wird dem Bankkonto gutgeschrieben
\item Privatkonto an Bank \\
Privatentnahme des Eigners, z.B. der Eigner begleicht in private Rechnung vom Firmenkonto
\item Bank an Zinserträge \\
Zinsen aus Kapitalanlagen werden auf dem Bankkonto gutgeschrieben
\item Verbindlichkeiten aus L\&L an Bank \\
begleichen eine Rechnung per Banküberweisung
\item Waren 20.000 an Bank 10.800 Verbindlichkeiten aus L\&L 9.200 \\
Kauf von Waren im Wert von 20.000 €. Davon werden 10.800 € per Banküberweisung gezahlt und 9.200 € auf Ziel gekauft 
\item Privatkonto an Fuhrpark \\
ein bisher betrieblich genutzter PKW wird vom Eigner entnommen und künftig privat genutzt
\item Verbindlichkeiten aus L\&L an Verbindlichkeiten ggü KI \\
ein Lieferant wandelt seine Forderungen in ein Darlehen
\item Kasse an Privatkonto \\
der Eigner legt Geld in die Kasse ein
\item Grundstücke und Bauten an Bank \\
Kauf von Grundstücken durch Banküberweisung
\item Kasse an Bank \\
Barabhebung vom Bankkonto
\item Betriebs- und Geschäftsausstattung an Verbindlichkeiten aus L\&L \\
Kauf von Büroeinrichtung auf Ziel
\item Betriebs- und Geschäftsausstattung an Privatkonto \\
der Eigner bringt bisher privat genutzt Büroausstattung (z.B. einen Schreibtisch) in das Unternehmen ein, die künftig betrieblich genutzt wird
\item Mietaufwand an Bank \\
Mietzahlung per Banküberweisung
\item Kasse an Forderungen aus L\&L \\
Kunde begleicht eine Rechnung durch Barzahlung
\item Forderungen aus L\&L an sonstige betriebliche Erträge(Provision) \\
ein (bebautes) Grundstück im Wert von 60.000 € wird veräußert. Der Käufer überweist 20.000 € auf das Bankkonto. Ihm werden 4.000 € als Kredit gewährt (zahlt auf Ziel) und es wird eine Hypothek in Höhe von 36.000 eingetragen
\item Verbindlichkeiten aus L\&L an Forderungen aus L\&L \\
Provisionserträge werden beim Großhändler eingefordert
\item Waren an Forderungen aus L\&L \\
eine Forderung wird an einen Lieferanten abgetreten
\item Büromaterial an Kasse \\
Kauf von Büromaterialien gegen Barzahlung
\item Verbindlichkeiten ggü KI X an Verbindlichkeiten ggü KI Y \\
Umschuldung: Darlehen bei Bank X wird durch Darlehen bei Bank Y abgelöst
\item sonstige betriebliche Aufwendungen an Kasse \\
ein Kassenfehlbetrag wird festgestellt
\item Betriebs- und Geschäftsausstattung an Waren \\
Tauschgeschäft: für Waren enthalten wir einen Bürostuhl
\item Forderungen aus L\&L an Waren \\
Verkauf von Waren auf Ziel
\item Verbindlichkeiten ggü KI an Fuhrpark \\
Verwendung eines PKW zur Begleichung einer Darlehensschuld
\end{enumerate} 
%-------------------------------------------
\paragraph{Aufgabe 7}
Welche Bilanzpositionen verändern sich durch die nachfolgenden Geschäftsvorfälle? Handelt es sich um einen Aktivtausch, einen Passivtausch, eine Bilanzverlängerung oder eine Bilanzverkürzung? Ist der Geschäftsvorfall erfolgswirksam? \\
\begin{enumerate}
\item Wareneinkauf gegen Barzahlung.
\item Warenverkauf unter Einkaufspreis gegen Banküberweisung.
\item Privatentnahme in bar.
\item Ein Kunde begleicht eine offene Rechnung per Banküberweisung.
\item Kauf eines PC per Banküberweisung.
\item Bareinzahlung bei der Bank (bei Bankschuld).
\item Überweisung der KFZ-Steuer für betrieblich genutzte PKW.
\item Der Eigentümer bringt einen privaten PKW zur betrieblichen Nutzung ein.
\item Warenverkauf über Einstandspreis auf Ziel
\item Die Hausbank reicht uns einen Kredit aus.
\item Die Bank zieht die Zinsen für den Kredit von unserem Konto ein.
\item Der Eigner überweist die Miete für seine Privatwohnung vom Bankkonto.
\item Löhne werden in bar ausgezahlt.
\item Die erste Tilgungsrate des Kredits wird an die Bank überwiesen.
\item Die Wartung eines Lieferwagens wird bar bezahlt.
\item Die betriebliche Stromrechnung wird per Banküberweisung beglichen.
\item Mietzahlungen für vermietete Büroräume gehen auf dem Bankkonto ein.
\end{enumerate}

\paragraph{Bilanzposition}
\begin{tabular}{|c|c|c|c|c|}
\hline
Nummer & + & - & Art & erfolgswirksam \\\hline
1 & Waren & Kasse & AT & nein \\
2 & Bank & EKK,Waren & AT/ AP- & ja \\
3 & & EKK,Kasse & AP- & nein\\
4 & Bank & Ford. aus LuL & AT & nein\\
5 & Betriebs- und GA & Bank & AT & nein \\
6 & & Kasse,Verb. ggü KI & AP- & nein \\
7 & & EKK,Bank & AP- & ja \\
8 & Fuhrpark, EKK & & AP+ & nein \\
9 & EKK, Ford. aus LuL & Waren & AP+/ AT & ja \\
10 & Bank, Verb. ggü KI & & AP+ & nein \\
11 & & Bank,EKK & AP- & ja \\
12 & & Bank,EKK & AP- & nein \\
13 & & EKK,Kasse & AP- & ja \\
14 & & Verb. ggü KI,Bank & AP- & nein \\
15 & & EKK,Kasse & AP- & ja \\
16 & & EKK,Bank & AP- & ja \\
17 & EKK, Bank & & AP+ & ja \\ \hline
\end{tabular}
%-------------------------------------------
\paragraph{Aufgabe 8}
Ihnen liegen die unten aufgeführte Eröffnungsbilanz eines Einzelkaufmanns und die Geschäftsvorfälle des Geschäftsjahres J1 vor. Alle anderen als die angesprochenen Steuerarten sind zu vernachlässigen. \\
\includegraphics[width=12cm,height=4cm]{Bilanz2.png} \\
Geschäftsvorfälle im Jahr J1:
\begin{enumerate}
\item Ein Kunde überweist zur Begleichung einer Rechnung 5.000 € auf das Bankkonto. 
\item Der Eigner bringt einen bisher privat genutzten PKW im Wert von 10.000 € zur betrieblichen Nutzung ein. 
\item Mietzahlungen in Höhe von 1.200 € werden per Banküberweisung vorgenommen. 
\item Mietzahlungen für untervermietete Räume in Höhe von 300 € gehen auf dem Bankkonto ein. 
\item Das Darlehen wird in Höhe einer Tilgungsrate von 5.000 € zurückgezahlt. Gleichzeitig überweisen wir der Bank Zinsen in Höhe von 2.500 €. 
\item Die Rechnung eines Lieferanten in Höhe von 3.000 € wird per Banküberweisung beglichen. 
\item Die Bank schreibt uns Zinsen in Höhe von 20 € gut. 
\item Briefmarken im Wert von 50 € werden bar bezahlt. 
\item Es werden Bürostühle im Wert von 1.100 € angeschafft und bar bezahlt. 
\item Gehälter in Höhe von 2.000 werden überwiesen. 
\item Die Steuer für die betrieblich genutzten PKW in Höhe von 2.700 € und die private Einkommensteuer des Eigners in Höhe von 6.000 € werden vom Firmenkonto überwiesen. 
\item Wir haben das Zahlungsziel eines Lieferanten überzogen und werden daher mit Verzugszinsen in Höhe von 25 € belastet. 
\item Aufgrund der erfolgreichen Vermittlung von Geschäften gehen 16.500 € Provisionen auf dem Bankkonto ein. 
\end{enumerate}
\begin{itemize}
\item[a)] Übernehmen Sie die Anfangsbestände direkt in die T-Konten, ohne die Buchungssätze für die Eröffnungsbuchungen zu notieren. Richten Sie erforderliche Erfolgskonten und gegebenenfalls ein Privatkonto ein. Bilden Sie die laufenden Buchungssätze für die angegebenen Geschäftsvorfälle im Jahr J1 und verdeutlichen Sie die Buchungen durch Eintragung in den T-Konten. 
\item[b)] Schließen Sie alle Konten ab und erstellen Sie die Schlussbilanz zum 31.12.J1. Die Abschlussbuchungen der Bestandskonten mit Ausnahme des Privat- und Eigenkapitalkontos sind nicht explizit vorzunehmen. 
\end{itemize}

a) \\
\textbf{Fuhrpark}
\begin{tabular}{cc|cc}
S & & & H \\\hline
AB & 50.000 & EB & 60.000 \\
 2) & 10.000 & & \\\hline
  & 60.000 & & 60.000 \\\hline
\end{tabular} \\
\textbf{EK-Konto}
\begin{tabular}{cc|cc}
S & & & H \\\hline
EB & 112.345 & AB & 100.000 \\
  & & A9 Gewinn & 8.345 \\
  & & A10 & 4.000  \\\hline
  & 112.345 & & 112.345  \\\hline
\end{tabular} \\
\textbf{Betriebs+GA}
\begin{tabular}{cc|cc}
S & & & H \\\hline
AB & 35.000 & EB & 35.100 \\
9) & 100 & & \\\hline
  & 35.100 & & 35.100 \\\hline
\end{tabular} \\
\textbf{Verbindl. ggü KI}
\begin{tabular}{cc|cc}
S & & & H \\\hline
5) & 5.000 & AB & 50.000 \\
EB & 45.000 & & \\\hline
  & 50.000 & & 50.000 \\\hline
\end{tabular} \\
\textbf{Ford. aus LuL}
\begin{tabular}{cc|cc}
S & & & H \\\hline
AB & 65.000 & 1) & 5.000 \\
  & & EB & 60.000 \\\hline
  & 65.000 & & 65.000 \\\hline
\end{tabular} \\
\textbf{Verb. aus LuL}
\begin{tabular}{cc|cc}
S & & & H \\\hline
6) & 3.000 & AB & 25.000 \\
EB & 22.025 & 12) & 25 \\\hline
  & 25.025 & & 25.025 \\\hline
\end{tabular} \\
\textbf{Bank}
\begin{tabular}{cc|cc}
S & & & H \\\hline
AB & 20.000 & EB & 19.420 \\
11) & 5.000 & 3) & 1.200 \\
4) & 300 & 5) & 7.500 \\
7) & 20 & 6) & 3.000 \\
13) & 16.500 & 10) & 2.000 \\
  & & 11) & 8.700 \\\hline
  & 41.820 & & 41.820 \\\hline
\end{tabular} \\
\textbf{Privatkonto}
\begin{tabular}{cc|cc}
S & & & H \\\hline
11) & 6.000 & 2) & 10.000 \\
A10) & 4.000 & & \\\hline
  & 10.000 & & 10.000 \\\hline
\end{tabular} \\
\textbf{Kasse}
\begin{tabular}{cc|cc}
S & & & H \\\hline
AB & 5.000 & 8) & 50 \\
  & & 9) & 100 \\
  & & EB & 4850 \\\hline
  & 5.000 & & 5.000 \\\hline
\end{tabular} \\

Buchungssätze für Geschäftsvorfälle im Jahr J1: 
\begin{enumerate}
\item Bank an Forderung aus LuL 5.000
\item Fuhrpark an Privatkonto 10.000
\item Mietaufwand an Bank 1.200
\item Bank an Mieterträge 300
\item Verbindung gegenüber KI (Darlehen) 5.000 an Bank 7.500 \\
Zinsaufwand 2.500 
\item Verbindung aus Lul an Bank 3.000
\item Bank an Zinserträge 20
\item Büromaterial an Kasse 50
\item BGAA an Kasse 100
\item Lohnaufwand und Gehalt an Bank 2.000
\item KFZ Steuer 2.700 an Bank 8.700 \\
Privatkonto 6.000
\item Zinsertrag an Verbindung aus LuL 25
\item Bank an sonstige Erlöse (Provisionen) 16.500
\end{enumerate}

\textbf{Mietaufwand}
\begin{tabular}{cc|cc}
S & & & H \\\hline
3) & 1.200 & A1 & 1.200 \\\hline
  & 1.200 & & 1.200 \\\hline
\end{tabular} \\
\textbf{Sonstige Erlöse (Provisionen)}
\begin{tabular}{cc|cc}
S & & & H \\\hline
A6 & 16.500 & 13) & 16.500 \\\hline
  & 16.500 & & 16.500 \\\hline
\end{tabular} \\
\textbf{Zinsaufwand}
\begin{tabular}{cc|cc}
S & & & H \\\hline
5) & 2.500 & A2 & 2525 \\
12) & 25 & & \\\hline
  & 2525 & & 2525 \\\hline
\end{tabular} \\
\textbf{Mieterträge}
\begin{tabular}{cc|cc}
S & & & H \\\hline
A7 & 300 & 4) & 300 \\\hline
  & 300 & & 300 \\\hline
\end{tabular} \\
\textbf{Lohn- und Gehaltsaufwand}
\begin{tabular}{cc|cc}
S & & & H \\\hline
  & & & \\
  & & & \\\hline
  & & & \\\hline
\end{tabular} \\
\textbf{Zinserträge}
\begin{tabular}{cc|cc}
S & & & H \\\hline
10) & 2.000 & A3 & 2.000 \\\hline
  & 2.000 & & 2.000 \\\hline
\end{tabular} \\
\textbf{KFZ Steuer}
\begin{tabular}{cc|cc}
S & & & H \\\hline
11) & 2.700 & A4 & 2.700 \\\hline
  & 2.700 & & 2.700 \\\hline
\end{tabular} \\
\textbf{Büromaterial}
\begin{tabular}{cc|cc}
S & & & H \\\hline
8) & 50 & A5 & 50 \\\hline
  & 50 & & 50 \\\hline
\end{tabular} \\
\textbf{GuV-Konto}
\begin{tabular}{cc|cc}
S & & & H \\\hline
A1 & 1.200 & A6 & 16.500 \\
A2 & 2.525 & A7 & 300 \\
A3 & 2.000 & A8 & 20 \\
A4 & 2.700 & & \\
A5 & 50 & & \\
A9 & 8.345 & & \\\hline
  & 16.820 & & 16.820 \\\hline
\end{tabular} \\

\begin{itemize}
\item[A1] GuV-Konto an Mietaufwand 1.200
\item[A2] GuV-Konto an Zinsaufwand 2.525
\item[A3] GuV-Konto an Lohn- und Gehaltsaufwand 2.000
\item[A4] GuV-Konto an KFZ-Steuer 2.700
\item[A5] GuV-Konto an Büromaterial 50
\item[A6] Sonsitge Erlöse an GuV-Konto 16.500
\item[A7] Mieterträge an GuV-Konto 300
\item[A8] Zinserträge an GuV-Konto 20
\item[A9] GuV-Konto an EKK 8.345
\item[A10] Privatkonto an EKK (Privateinlage) 4.000
\end{itemize}

b) \\
Schlussbilanz zum 31.12.J1 in Euro $€$ \\
\begin{tabular}{cc|cc}
A & & & P \\\hline
Fuhrpark & 60.000 & EKK & 112.345 \\
Betreibs+GA  & 35.100 & Verb. ggü KI & 45.000 \\
Ford. aus LuL  & 60.000 & Verb. aus LuL & 22.025 \\
Bank & 19.420 & & \\
Kasse & 4.850 & & \\\hline
  & 179.370 & & 179.370 \\\hline
\end{tabular}

Gewinn: EB-AB-Einlagen+Entnahme \\
$112.345-100.000-10.000+6.000 = 8345$
%----------------------------------------------------------------------------------------
\section{Laufenden Geschäftsvorfälle}
%------------------------------------------------------
\subsection{Warenverkehr, Materialverbrauch, Erzeugnisbestände}
%-------------------------------------------
\paragraph{Aufgabe 1}
\begin{itemize}
\item[a)] Buchen Sie die unten aufgeführten Geschäftsvorfälle unter Verwendung getrennter Wareneinkaufs- und Warenverkaufskonten. Die Umsatzsteuer ist zu vernachlässigen. Der Warenanfangsbestand beträgt 75.000 €. \\
 \begin{enumerate}
 \item Verkauf von Waren auf Ziel im Wert von 10.000 € laut Ausgangsrechnung.  
 \item Zahlung von Kreditzinsen in Höhe von 1.200 € per Banküberweisung.  
 \item Kauf von Waren gegen Barzahlung im Wert von 2.500 € laut Eingangsrechnung.  
 \item Der Eigentümer entnimmt 1.000 € aus der Kasse.  
 \item Zielkauf von Waren im Wert von 20.000 € laut Eingangsrechnung.  
 \item Der Kunde aus Geschäftsvorfall 1. begleicht seine Rechnung per Banküberweisung.  
 \item Verkauf von Waren im Wert von 17.000 € laut Ausgangsrechnung. Zum Ausgleich übernimmt der Abnehmer eine Verbindlichkeit aus L\&L.  
 \item Das Gehalt des Geschäftsführers (6.000 €) wird per Banküberweisung ausgezahlt.  
 \item Überweisung der Rechnung aus Vorfall 5. vom Bankkonto.  
 \item Barkauf von Waren im Wert von 5.000 € laut Eingangsrechnung.
 \end{enumerate}
\item[b)] Schließen Sie die Warenkonten nach der Nettomethode ab. Der Endbestand laut Inventur beträgt 70.000 €. Stellen Sie das Wareneinkaufs-, Warenverkaufs- und GuV-Konto in TKonten-Form dar. 
\item[c)] Schließen Sie die Warenkonten nach der Bruttomethode ab. Der Endbestand laut Inventur beträgt 70.000 €. Stellen Sie das Wareneinkaufs-, Warenverkaufs- und GuV-Konto in TKonten-Form dar. 
\item[d)] Erläutern Sie den Vorteil der Brutto- gegenüber der Nettomethode beim Abschluss getrennter Warenkonten.
\end{itemize}

a) 
\begin{enumerate}
\item Ford. aus LuL an Warenverkauf 10.000
\item Verb. ggü KI an Bank 1200
\item Wareneinkauf an Kasse 2500
\item PK an Kasse 1000
\item Wareneinkauf an Verb. ggü LuL 20.000
\item Bank an Ford. aus LuL 10.000
\item Verb. aus LuL Warenverkauf 17.000
\item Lohn- und Gehaltsaufwand an Bank 6.000
\item Verb. aus LuL an Bank 20.000
\item Wareneinkauf an Kasse 5.000
\end{enumerate}

d) \\
Vorteile: Nur bei der Bruttomethode wird die Zusammensetzung des Warenerfolgs oder Verlusts deutlich. Sowohl bei der Netto, als auch bei der Bruttomethode wird der Erfolg in gleicher Höhe ausgewiesen (hier: Verlust in Höhe von 12.700)
%-------------------------------------------
\paragraph{Aufgabe 2}
\begin{itemize}
\item[a)] Buchen Sie die unten aufgeführten Geschäftsvorfälle. Gehen Sie generell von einem Umsatzsteuersatz in Höhe von $19 \%$ aus. \\
  \begin{enumerate}
  \item Barkauf von Waren zum Bruttowert von 7.140 €.   \\
  Wareneinkauf 6.000 (7.140/ 1,19) an Kasse 7140 \\
  Vst 1140 (6000*0,19)
  \item Barkauf von Büromaterial (Toner) im Wert von 300 € (netto). \\
  Büromaterial 300 an Kasse 357 \\
  Vst 57  
  \item Kauf eines Laptops per Banküberweisung im Wert von 500 € zuzüglich USt. \\
  BGAA 500 an Bank 595 \\
  Vst 95  
  \item Waren im Wert von 2.000 € (netto) werden auf Ziel verkauft. \\
  Forderung aus L \& L 2380 an Warenverkauf 2000 \\
  Ber. Ust 380  
  \item Wir verkaufen Waren (bar) im Wert von 3.570 € inklusive USt. \\
  Kasse 3570 an Warenverkauf 3000 \\
  Ber. Ust 570
  \item Ein Kunde begleicht eine Rechnung. Er überweist 4.760 €. \\
  Bank 4760 an Forderung L \& L 4760  
  \item Barkauf von Schreibtischen. Die Umsatzsteuer beträgt 190 €. \\
  BGAA 1000 an Kasse 1190 \\
  Vst 190 
  \end{enumerate}
\item[b)] Ermitteln Sie die Umsatzsteuerzahllast nach der Zwei-Konten-Methode. Stellen Sie die Steuerkonten in T-Konten-Form dar. Buchen Sie auch die Steuerzahlung bzw. eine Erstattung durch das Finanzamt per Banküberweisung. 
\item[c)] Ermitteln Sie die Umsatzsteuerzahllast nach der Drei-Konten-Methode. Stellen Sie die Steuerkonten in T-Konten-Form dar. Buchen Sie auch die Zahlung bzw. eine Erstattung durch das Finanzamt per Banküberweisung. 
\item[d)] Geben Sie die Buchungssätze für die Fälle unter b) und c) an, wenn der Voranmeldezeitraum zum Bilanzstichtag endet und die Steuerzahlung bzw. eine Erstattung erst im neuen Jahr stattfindet. 
\end{itemize}

%-------------------------------------------
\paragraph{Aufgabe 3}
Bilden Sie die Buchungssätze für die nachstehend aufgeführten Geschäftsvorfälle. Gehen Sie von einem Umsatzsteuersatz von $10 \%$ aus. Skonti sind nach der Bruttomethode zu buchen. \\
\begin{enumerate}
\item Ein Kunde begleicht eine Rechnung durch Banküberweisung. Er zieht $3 \%$ Skonto ab. Der  (Brutto-) Rechnungsbetrag lautet über 4.400 €. \\

Bank 4.268 an Forderung aus L\&L 4400 € \\
Skontoaufwand 120 €
bet. Ust 12 €
\item Gewerbesteuer in Höhe von 800 € und Löhne in Höhe von 5.000 € werden per Banküberweisung gezahlt. \\

Gewerbesteuer 800 an Bank 5800 \\
L\&G-Aufwand 5000 €
\item Wir kaufen Waren im Wert von 16.500 € (brutto) auf Ziel. Bei Zahlung innerhalb von 10 Tagen ist $2\%$ Skonto abzuziehen. \\

Wareneinkauf 15.000 an Verbindlichkeinten aus L\&L 16.500
\item Die Waren aus Vorfall 3. werden am nächsten Tag per Banküberweisung bezahlt. \\

Verb. aus L\&L 16.500 an Bank 16.170, Skontoerträge 300. Vorsteuer 30
\item Die Tochter des Eigners entnimmt Waren mit einem Wert von 500 € im Entnahmezeitpunkt aus dem Lager, um eine private Party zu veranstalten. Die Anschaffungsnebenkosten betragen 20 €. Der Verkaufspreis der Waren beträgt 750 €. Auf der Party schüttet sie Rotwein über ihr neues Kleid, das daraufhin gereinigt werden muss. \\

Wareneinkauf an Bezugsaufwand 20 € \\
Privatkonto 572 € an Eigenverbrauch 520 € \\
ber. Ust 52 \\
Werteinsatz am Privatkonto $\Rightarrow$ Eingeschäftskosten
\item Der Eigner entnimmt 1.000 € aus der Kasse. \\

Privatkonto an Kasse 1000 €
\item Wir verkaufen Waren im Wert von 2.000 € (netto) auf Ziel. Der Kunde ist berechtigt, bei Zahlung innerhalb von 10 Tagen $4 \%$ Skonto abzuziehen. \\

Foderung aus L\&L 2200 an Warenverkauf 20.000 \\
ber. Ust 200
\item Der Kunde begleicht die Rechnung aus Vorfall 7. drei Wochen später per Banküberweisung. \\

Banl an Forderung aus L\&L 2200 \\
falls Bezahlung innerhalb des 10-Tagen-Zeit: Bank 2112 and For. aus L\&L 2200\\
Skontoaufwand 80
\item Wir kaufen eine Spezialmaschine im Wert von 100.000 € zuzüglich USt auf Ziel. Die Installation der Maschine kostet 22.000 € (brutto). Für den Betrieb der Maschine werden außerdem Schmiermittel für zukünftige Schmierungen i.W.v. 100 € (brutto) gegen Barzahlung beschafft. 
\item Wir schicken mangelhafte, auf Ziel gekaufte Ware im Wert von 10.000 € (netto) zurück.  \\

Verb. aus L\&L 11.000 an Wareneinkauf 10.000, Vorsteuer 1.000
\item Zinsen in Höhe von 150 € werden auf dem betrieblichen Konto gutgeschrieben. \\

Bank an Zinserträge 150 €
\item Zinsen aus einer privaten Geldanlage des Eigners in Höhe von 75 € werden auf dem betrieblichen Konto gutgeschrieben. \\

Bank an Privatkonto 75 €
\item Die Kosten für das Hotel auf einem privaten Wochenendausflug des Eigners in Höhe von 330 € (brutto) werden vom betrieblichen Bankkonto abgebucht. \\

Privatkonto an Bank 330 €
\item Aus der Kasse werden 220 € entnommen, um Spesen (brutto) auf einer Dienstreise zu begleichen. \\

Reisekosten 200 an Kasse 220 € \\
(Aufwandkonto) \\
Vorsteuer 20
\item Ein neuer Transporter mit einem Nettopreis in Höhe von 25.000 € wird bestellt. Wir leisten ohne Anzahlungsrechnung eine Anzahlung in Höhe von 13.200 € per Banküberweisung. \\

gleistete Anzahlung an Banl 13.200 \\
Bestellung aber keine Anzahlungsrechnung \\
$\rightarrow$ keine Vorsteuer - Abzug möglich
\item Der Transporter aus Vorfall 15. wird einen Monat später geliefert. Der Restbetrag wird sofort unter Abzug von $6 \%$ Skonto per Banküberweisung beglichen. \\

Fuhrpark 25.000 an Verb. aus L\&L 27.500 \\
Vorsteuer 2.500 \\
Verb. aus L\&L an geleistete Anzahlung 13.800 \\
Verb. aus L\&L 14.300 an Bank 13.442, Skontoertrag 780, Vorsteuer 78 \\
Skontoertrag: 143* 0,06 / 1,1 = 780

\item Wir verkaufen Waren auf Ziel. Der Warenwert beträgt 10.000 €. Da es sich um Restposten handelt und das Lager geräumt werden soll, wird dem Kunden ein Rabatt von $25 \%$ eingeräumt. \\

Forderung aus L\&L 8.750 an Warenverkauf 7.500, ber. Ust 750
\item Ein Kunde begleicht eine Rechnung unter Abzug von $2 \%$ Skonto. Er überweist 1.078 €.  \\

Bank 1078 an Forderung aus L\&L 11.000 \\
Skontoaufwand 20 \\
ber. Ust 2
\item Für den geplanten Bau eines Firmenparkplatzes wird ein unbebautes Grundstück erworben. Das Grundstück kostet 500.000 €. Der Kauf wird durch Banküberweisung von 400.000 € und durch die Übernahme einer auf dem Grundstück liegenden Hypothek in Höhe von 100.000 € finanziert. Durch den Erwerb fallen zusätzliche Kosten an: Der Eintrag ins Grundbuch (von der Umsatzsteuer befreit) kostet 1.000 €. Die Maklerprovision beträgt 25.000 € (netto). Außerdem ist Grunderwerbsteuer (von der Umsatzsteuer befreit) von $3,5 \%$ auf den Kaufpreis zu entrichten. Diese Zahlungsverpflichtungen werden per Banküberweisung beglichen. \\

Grund+Bauten 500.000 an Bank 400.000, Verb. (Hypothek) 100.000 \\
Grund+Bauten 43.500 (= 1000+25.000+500.000 * 0,035)  an Bank 46.000 \\
Vorsteuer 2500 (= 25.000 * 0,1)
\item Wir verkaufen Waren auf Ziel. Der vom Kunden zu überweisende Betrag auf der Rechnung lautet auf 24.530 €. \\
\begin{itemize}
\item[a)] Dem Kunden wurde ein Rabatt in Höhe von $20 \%$ gewährt. Wie hoch war der Warenwert (netto) ursprünglich? \\

$24.530 \overset{\wedge}{=} 80 \%$ \\
$24.530/0,8 = 30.662,5 \overset{\wedge}{=} 100 \%$ \\
$30.662,5/1,1 = 27.875$ \\
Somit liegt der netto Warenwert bei 27.875 €.
\item[b)] Der Kunde stellt nachträglich fest, dass $10 \%$ der Waren verdorben sind. Die Rücksendung führt zu einer Gutschriftsanzeige. \\

Warenverkauf 2.230 an Forderung aus L\&L 2.453 \\
ber. Ust 223
\item[c)] Da der Kunde durch seinen Einkauf eine Umsatzgrenze überschritten hat, gewähren wir ihm nachträglich einen Bonus in Höhe von 5.000 € (netto). Es erfolgt eine Verrechnung mit einer Forderung.  \\

Kundenboni 5.000 an Forderung aus L\&L 5.500 \\
ber. Ust 500
\end{itemize}
\item Ein Kunde schickt mangelhafte Ware zurück, was zu einer Korrektur der Umsatzsteuer in Höhe von 240 € führt.  \\
Warenverkauf 2400 an Foerderung aus L\&L 2640 \\
ber. Ust 240
\item Eine Großkunde zahlt 28.600 € per Banküberweisung für eine Warenbestellung an. 
\item Aufgrund unserer großen Abnahmemenge während des abgelaufenen Jahres gewährt uns ein Zulieferer einen nachträglichen Bonus in Höhe von 10.000 € (netto), der mit Verbindlichkeiten verrechnet wird. 
\item Wir kaufen Waren im Wert von 20.000 €. Die Zahlungsbedingungen lauten: 20 Tage: $2 \%$ Skonto, 30 Tage: netto. Nach fünf Tagen überweisen wir 12.936 € an den Lieferanten. Der Rest wird genau 30 Tage nach Rechnungsstellung überwiesen. 
\item Wir kaufen Waren im Nettowert von 40.000 € auf Ziel. Zusätzlich werden uns Europaletten mit 1.000 € netto in Rechnung gestellt. Im Fall einer späteren Rücksendung der Paletten würden $90 \%$ ihres Netto-Rechnungsbetrags gutgeschrieben werden. 
\item Wir senden Waren im Wert von 4.000 € aus Vorfall 25. wegen Mängeln zurück und erhalten eine Gutschriftsanzeige. Außerdem schicken wir die Europaletten zurück. 
\item Es wird festgestellt, dass die Waren aus den Vorfällen 3. und 4. zu $20 \%$ ungenießbar sind. Dieser Teil wird an den Lieferanten zurückgeschickt, der uns eine Gutschrift überweist
\end{enumerate}
%-------------------------------------------
\paragraph{Aufgabe 4}
Buchen Sie folgende Geschäftsvorfälle. Der Umsatzsteuersatz beträgt einheitlich $19 \%$. Die Erfassung des Materialverbrauchs soll nach der Skontrationsmethode erfolgen. \\
\begin{enumerate}
\item Wir kaufen Rohstoffe im Wert von 7.140 € (brutto) ein. Die Rechnung wird sofort per Banküberweisung beglichen.  \\

RHB-Stoffe 6.000 an Bank 7.140 \\
Vst 1.140
\item Laut Materialentnahmeschein werden Rohstoffe im Wert von 1.000 € aus dem Lager entnommen. Aus 10 Einheiten des Rohstoffs kann immer genau ein Stück des Endprodukts hergestellt werden. \\
Aufwand für RHB Stoffe an RHB Stoffe 1.000 
\item Fertigerzeugnisse im Wert von 3.000 € (netto) werden auf Ziel verkauft. \\
Forderung aus L \& L 3570 an Umsatzerlöse 3.000 \\
ber. Ust 570 
\item Für eine gemietete Spezialmaschine wird eine Leasingrate in Höhe von 595 € (brutto) an den Leasinggeber überwiesen. \\
Leasingaufwand 500 an Bank 595 \\
Vst 95
\item Der Bestand an Fertigerzeugnissen hat sich in der betrachteten Periode um 2.000 € erhöht. \\
Fertige Erzeugnisse an Bestandsveränderung fertige Erzeugnisse 2.000
\end{enumerate}

Skontrationsmethode: laufende Erfassung der Abgänge \\
Inventurmethode: EB über Inventur ermittelt \\
Verbrauch: alle Abgänge aufsummiert (Abgang 1+ Abgang 2+ ...) \\
Umsatzsteuersatz von $19 \%$
%-------------------------------------------
\paragraph{Aufgabe 5}
Aus der Eröffnungsbilanz eines Industriebetriebs werden die nachfolgenden Konten entnommen:  \\

\begin{tabular}{|c|cc|}
\hline
Konto & Soll & Haben \\\hline
Rohstoffe & 100.000 & -- \\
Hilfsstoffe & 50.000 & -- \\
Betriebsstoffe & 25.000 & -- \\\hline
\end{tabular} \\

Während der betrachteten Abrechnungsperiode werden Rohstoffe im Wert von 23.800 € (inkl. $19 \%$ USt) beschafft. Ermitteln Sie den Materialverbrauch, schließen Sie die relevanten Bestandskonten und das Aufwandskonto und buchen Sie den Verbrauch an RHB-Stoffen. Buchen Sie nach der Inventurmethode. Die Schlussbestände laut Inventur betragen: Rohstoffe 55.000 €, Hilfsstoffe 30.000 € und Betriebsstoffe 7.000 €. \\

\textbf{Rohstoffe} 
\begin{tabular}{cc|cc}
S & & & H \\\hline
AB  & 100.000 & EB 1) & 55.000 \\
Zugänge  & 20.000 & 4) & 65.000 \\\hline
  & 120.000 & & 120.000 \\\hline
\end{tabular} \\
\textbf{Hilfsstoffe}
\begin{tabular}{cc|cc}
S & & & H \\\hline
AB & 50.000 & EB 2) & 30.000 \\
  & & 5) & 20.000 \\\hline
  & 50.000 & & 50.000 \\\hline
\end{tabular} \\
\textbf{Betriebsstoffe}
\begin{tabular}{cc|cc}
S & & & H \\\hline
AB & 25.000 & EB 7) & 7.000 \\
  & & 6) & 18.000 \\\hline
  & 25.000 & & 25.000 \\\hline
\end{tabular} \\
\textbf{Aufwand für RHB}
\begin{tabular}{cc|cc}
S & & & H \\\hline
4) & 65.000 & 7) & 103.000 \\
5) & 20.000 & & \\
6) & 18.000 & & \\\hline
 & 103.000 & & 103.000 \\\hline
\end{tabular} \\
\textbf{GuV-Konto}
\begin{tabular}{cc|cc}
S & & & H \\\hline
7) & 103.000 & & \\
  & & & \\
\end{tabular} \\
\textbf{SBK}
\begin{tabular}{cc|cc}
S & & & H \\\hline
1) & 55.000 & & \\
2) & 30.000 & & \\
3) & 7.000 & & \\
  & & & \\
\end{tabular} \\

\textbf{Endbestand}:
\begin{enumerate}
\item SBK an Rohstoffe 55.000
\item SBK an Hilfsstoffe 30.000
\item SBK an Betriebsstoffe 7.000
\item Aufwand für RHB an Rohstoffe 65.000
\item Aufwand für RHB an Hilfsstoffe 20.000
\item Aufwand für RHB an Betriebsstoffe 18.000
\item GuV Konto an Aufwand für RHB 103.000
\end{enumerate}
%-------------------------------------------
\paragraph{Aufgabe 6}
Das Unternehmen N.Otebook OHG erstellt im Jahr X1 Fertigerzeugnisse, wobei für Zwecke der Rechnungslegung folgendes aufbereitetes Datenmaterial (in €) vorliegt: \\
\includegraphics[width=10cm,height=6cm]{Datenmaterial.png} 
\begin{itemize}
\item[a)] Berechnen Sie anhand dieser Angaben den höchst- bzw. geringstmöglichen Bilanzansatz nach HGB für obige Fertigerzeugnisse. \\

\begin{tabular}{|c|cc|c|c|}
\hline
 & & & Untergrenze & Obergrenze \\\hline
Pflichtposition & Einzelkosten & Fertigungslöhne 2) & 30.000 & \\
 & & Fertigungsmaterial 6) & 15.000 & \\
 & & Sondereinzelkosten der Fertigung 3) & 20.000 & \\
 & Gemeinkosten & Materialgemeinkosten 7) & 3.000 & \\
 & & Fertigungsgemeinkosten 4)+5) & 17.000 & \\
 & & Anteilige Abschreibung 1) & 3.000 & \\\hline
 & & & 88.000 & \\\hline
Wahlposition & & Verwaltungsgemeinkosten 12) & & 5.000 \\
 & & Fremdkapitalzinsen 9) & & 8.000 \\
 & & Aufwendungen für Altersvorsorge 11) & & 9.000 \\\hline
 & & & & 110.000 \\\hline 
\end{tabular}
\item[b)] Wie sind die Fertigerzeugnisse anzusetzen, um dem Ziel eines möglichst hohen/niedrigen Jahresüberschusses in einem Jahr mit Lageraufbau (Produktion > Absatz) gerecht zu werden? \\

\underline{Hoher Jahresüberschuss} durch möglichst hohe Aktivierung der Aufwendungen, deshalb Wahl der Obergrenze \\
Höhere Bestandserhöhung: mehr Erträge bei Lageraufbau \\
$\rightarrow$ Aufwendungen werden in Herstellungskosten \\

\underline{Niedriger Jahresüberschuss} durch möglichst niedrige Aktivierung der Aufwendungen \\
$\rightarrow$ Wahl der Untergrenze möglichst viele Aufwendungen in der GuV erfasst
\item[c)] Welchen Effekt hat die Entscheidung aus b), wenn im folgenden Geschäftsjahr ein Lagerabbau (Produktion < Absatz) stattfindet? \\

Je höher die Bestandserhöhung in der Vergangenheit bewertet wurde (je mehr Wahlpositionen mit einbezogen wurden), desto höher fallen die Bestandsminderungen in Folgeperioden bei Lagerabbau aus. \\
Der Lagerabbau wird mit denselben Herstellungskosten bewertet wie zuvor der Lageraufbau.
\end{itemize}
%-------------------------------------------
\paragraph{Aufgabe 7}
Das Konto „Roh-, Hilfs- und Betriebsstoffe“ in einem Industriebetrieb weist zum Jahresanfang einen Stand in Höhe von 150.000 € auf. Während des Geschäftsjahres wurden keine Stoffe angeschafft. Die Anfangsbestände auf den Konten „unfertige Erzeugnisse“ und „fertige Erzeugnisse“ lauten 40.000 € und 27.000 €. Laut Inventur ergeben sich folgende Endbestände: \\
\begin{tabular}{cc}
Roh-, Hilfs- und Betriebsstoffe: & 77.000 € \\
Unfertige Erzeugnisse: &  18.300 € \\
Fertige Erzeugnisse:  &  31.500 € 
\end{tabular} \\

\begin{itemize}
\item[a)] Ermitteln Sie den Materialverbrauch und die Bestandsveränderungen, schließen Sie die Bestandskonten und das Aufwandskonto und buchen Sie den Verbrauch an RHB-Stoffen. Führen Sie die Jahresabschlussbuchungen nach dem Gesamtkostenverfahren und der Inventurmethode durch. Verwenden Sie lediglich ein Sammelkonto für die Bestandsveränderungen.
\item[b)] Wie hoch ist der Periodenerfolg, wenn im betrachteten Jahr Umsatzerlöse in Höhe von 100.000 € erwirtschaftet wurden? \\

Periodenerfolg: 100.000 - (73.000+17.200)= 9.800
\item[c)] Wie wirken sich Bestandserhöhungen und Bestandsminderungen auf den Periodenerfolg aus? \\

\underline{Bestandserhöhungen} wirken sich ergebniserhöhend aus: im Gesamtkostenverfahren explizit ausgewiesen (Bestandveränderung), im Umsatzkostenverfahren durch niedrige Umsatzaufwendungen berücksichtigt. \\

\underline{Bestandsverminderung} (hier der Fall) wirken sich ergebnismindernd aus.
\end{itemize}

a) \\
unfertige Erzeugnisse uE \\
fertige Erzeugnisse fE \\

\textbf{RHB-Stoffe}
\begin{tabular}{cc|cc}
S & & & H \\\hline
AB & 150.000 & EB & 77.000 \\
  & & 4) & 73.000 \\\hline
  & 150.000 & & 150.000 \\\hline
\end{tabular} \\
\textbf{Aufwand für RHB}
\begin{tabular}{cc|cc}
S & & & H \\\hline
4) & 73.000 & 7) & 73.000 \\\hline
  & 73.000 & & 73.000 \\\hline
\end{tabular} \\
\textbf{Unfertige Erzeugnisse}
\begin{tabular}{cc|cc}
S & & & H \\\hline
AB & 40.000 & EB & 18.300 \\
  & & 5) & 21.700 \\\hline
  & 40.000 & & 40.000 \\\hline
\end{tabular} \\
\textbf{Bestandsveränderung fE}
\begin{tabular}{cc|cc}
S & & & H \\\hline
5) & 21.700 & 6) & 4.500 \\
  & & 8) & 17.200 \\\hline
  & 21.700 & & 21.700 \\\hline
\end{tabular} \\
\textbf{fertige Erzeugnisse}
\begin{tabular}{cc|cc}
S & & & H \\\hline
AB & 27.000 & EB & 31.500 \\
6) & 4.500 & & \\\hline
  & 31.500 & & 31.500 \\\hline
\end{tabular} \\
\textbf{GuV Konto}
\begin{tabular}{cc|cc}
S & & & H \\\hline
7) & 73.000 & & \\
8) & 17.200 & & \\\hline
\end{tabular} \\
\textbf{SBK}
\begin{tabular}{cc|cc}
S & & & H \\\hline
1) & 77.000 & & \\
2) & 18.300 & & \\
3) & 31.500 & & \\\hline
\end{tabular} \\

\begin{enumerate}
\item SBK an RHB Stoffe 77.000
\item SBK an uE 18.300
\item SBK an fE 31.500
\item Aufwand für RHB an RHB Stoffe 73.000
\item Bestandsveränderung fE an uE 21.700
\item fE an Bestandsänderung fE 4.500
\item GuV Konto an Aufwand RHB 73.000
\item GuV Konto an Bestandsveränderung fE 17.200
\end{enumerate}
%-------------------------------------------
\paragraph{Aufgabe 8}
Die Shine OHG produziert in einem einstufigen Fertigungsprozess Solarzellen. Für das Jahr J1 ergibt sich folgende Situation. Die Umsatzsteuer bleibt unberücksichtigt: \\
\includegraphics[width=9cm,height=5cm]{Fertigungsprozess.png} 
\begin{itemize}
\item[a)] Ermitteln Sie den Endbestand an Fertigerzeugnissen.
\item[b)] Geben Sie alle im Zusammenhang mit dem Herstellungsprozess anfallenden Buchungssätze an (inkl. Abschlussbuchungen der angesprochenen Konten mit Ausnahme des GuV-Kontos), wenn das Gesamtkostenverfahren Anwendung findet. Stellen Sie die wertmäßigen Bewegungen zusätzlich auch in Kontenform dar.
\item[c)] Vgl. b), jedoch nach dem Umsatzkostenverfahren.
\end{itemize}
%-------------------------------------------
\paragraph{Aufgabe 9}
Die Firma „Radiopherm“ stellt seit 2015 ein generisches Kopfschmerzmittel für den deutschen Markt her. Folgende Daten über die Produktion wurden für das Jahr 2016 und das Vorjahr 2015 erfasst: \\

\begin{tabular}{|cc|c|c|}
\hline
& & \textbf{2015} & \textbf{2016} \\\hline
Hergestellte Menge & [Stück] & 10.000.000 & 12.000.000  \\\hline 
Abgesetzte Menge & [Stück] & 6.000.000 & 16.000.000  \\\hline 
Verbrauch RHB-Stoffe & [€] & 150.000 & 180.000  \\\hline 
Fertigungslöhne & [€] & 80.000 & 120.000  \\\hline 
Aufwendungen für Altersversorgung & [€] & 20.000 & 24.000  \\\hline 
Gehalt für Fr Maier (Buchhaltung) & [€] & 20.000 & 24.000  \\\hline 
Werbemaßnahmen & [€] & 15.000 & 10.000  \\\hline 
\end{tabular} \\

Radiopherm kauft die benötigte Menge an Roh-, Hilfs- und Betriebsstoffen. Auszahlungen werden über das Bankkonto der Firma getätigt. Bei den allgemeinen Verwaltungskosten handelt es sich um das Gehalt der Buchhalterin Frau Maier. Frau Maier ist nicht in den Fertigungsprozess des neuen Kopfschmerzmittels einbezogen. Die Werbemaßnahmen erfolgen in Form von Barzahlungen an ausgewählte Ärzte, um deren Entscheidungsfindung bei der Verschreibung von Arzneimitteln zu unterstützen. Der Verkaufspreis je Stück beträgt 0,3 €. Zu Beginn des Jahres 2015 beträgt der Anfangsbestand der Kasse 50.000 €, des Bankkontos 500.000 € und des Eigenkapitals 550.000 €. Die Aufwendungen für Altersversorgung der im Fertigungsprozess eingesetzten Mitarbeiter werden direkt an eine externe Pensionskasse per Bank überwiesen. Die Umsatzsteuer ist zu vernachlässigen. \\
Für das Geschäftsjahr 2015 ist die Firma „Radiopherm“ am Ausweis eines möglichst geringen Gewinns interessiert. \\
\begin{itemize}
\item[a)] Nach dem Gesamtkostenverfahren 
\begin{itemize}
\item[a1] Buchen Sie alle Geschäftsvorfälle im Zusammenhang mit Herstellung und Verkauf in den Jahren 2015 und 2016 (inklusive Abschlussbuchungen der angesprochenen Konten).
\item[a2] Buchen Sie das Jahr 2015 auf T-Konten
\item[a3] Geben Sie jeweils den Jahresüberschuss für 2015 und 2016 an.
\end{itemize}
\item[b)] Vergleiche a), jedoch nach dem Umsatzkostenverfahren.
\item[c)] Wie verändert sich der Jahresüberschuss 2015 und 2016, wenn 2015 ein möglichst hoher Gewinn ausgewiesen werden soll? Berechnen Sie die veränderten Jahresüberschüsse für 2015 und 2016. Welche Schlussfolgerung lässt sich daraus in Bezug auf die Spielräume von Bilanzpolitik ziehen? 
\end{itemize}
%------------------------------------------------------
\subsection{Lohn und Gehalt}
%-------------------------------------------
\paragraph{Aufgabe 10}
Buchen Sie folgende Geschäftsvorfälle. Der Solidaritätszuschlag ist zu vernachlässigen. \\
\begin{enumerate}
\item Es werden Löhne an Arbeiter in der Produktion in Höhe von 10.000 € (netto) überwiesen. Der Arbeitgeber-Anteil zur Sozialversicherung beträgt 3.400 € Lohn- und Kirchensteuer fallen in Höhe von 3.100 € an. Buchen Sie auch die Begleichung der abzuführenden Abgaben.
\item Für die Putzkolonne überweisen wir an eine Leiharbeitsfirma den monatlichen Betrag in Höhe von 8.925 € inklusive $19 \%$ Umsatzsteuer. 
\item Für die Werbung neuer Mitarbeiter fallen 300 € (zzgl. $19 \%$ USt) an Kosten für eine Internetanzeige an, die an den Verlag überwiesen werden. 
\item Für Aushilfskräfte („450 Euro- Minijobs“) fallen Löhne in Höhe von 1.350 € an, die bar ausgezahlt werden. Es findet die $2 \%$-Pauschbesteuerung Anwendung. Der Arbeitgeber entrichtet Pauschalbeiträge von insgesamt 424,17 € (Lohnsteuer 27 € und Versicherungsbeiträge 397,17 €) an die Deutsche Rentenversicherung Knappschaft-Bahn-See, die voll vom Arbeitgeber zu tragen sind. Buchen Sie auch die Begleichung der abzuführenden Abgaben.
\item Dem neuen Mitarbeiter H. Abgier wird am 15.06. ein Gehaltsvorschuss in Höhe von 3.000 € überwiesen. Dieser Vorschuss soll in drei gleich hohen Raten mit den monatlichen Bezügen verrechnet werden. Am Monatsende bekommt Herr Abgier sein Gehalt überwiesen. Er verdient ein Brutto-Monatsgehalt in Höhe von 4.150 €. An Lohnsteuer fallen 930 € an. Die Sozialversicherungsbeiträge erreichen zusammen (Arbeitgeber- und Arbeitnehmer-Anteil) eine Höhe von 1.640 €. Buchen Sie den Gehaltsvorschuss, sowie die Gehaltsabrechnung und die Begleichung der abzuführenden Abgaben. 
\item Am Ende des Monats November liegen für die Gehaltsabrechnung folgende Daten vor:  
\begin{itemize}
 \item Bruttogehälter: 25.000 €   
 \item Lohnsteuer: 6.000 €   
 \item Kirchensteuer: 450 €   
 \item Arbeitgeber-Anteil zur Sozialversicherung: 4.600 € 
\end{itemize} 
Im betrachteten Monat haben die Mitarbeiter Waren im Warenwert (netto) von 800 € bezogen (werden mit den Bruttogehältern verrechnet; der Umsatzsteuersatz beträgt $19 \%$). Buchen Sie die vollständige Verrechnung mit dem Nettogehalt und die Begleichung der abzuführenden Abgaben. 
\item Die Mitarbeiterin Mia Sparsam bezieht ein Brutto-Monatsgehalt in Höhe von 1.750 €. Sie zahlt monatlich 25 € vermögenswirksame Leistungen in einen Sparplan ein. Wir haben uns vertraglich dazu verpflichtet, weitere 10 € pro Monat für die Vermögensbildung von Frau Sparsam beizusteuern. Ansonsten fallen monatlich Lohnsteuer in Höhe von 180 € sowie der Arbeitgeber-Anteil zur Sozialversicherung in Höhe von 350 € an. Buchen Sie die Gehaltsabrechnung für den Monat September, sowie die Begleichung der abzuführenden Abgaben.
\end{enumerate}
%----------------------------------------------------------------------------------------
\section{Vorbereitende Abschlussbuchungen}
%------------------------------------------------------
\subsection{Anlagevermögen}
%-------------------------------------------
\paragraph{Aufgabe 1}
Bilden Sie die Buchungssätze für nachfolgende Geschäftsvorfälle: \\
\begin{enumerate}
\item Am 01.01.J1 (erster Januar des Jahres 1) wird eine Maschine für 500.000 € (netto) erworben. Außerdem fallen Transportkosten in Höhe von 50.000 € (netto) an. Der Umsatzsteuersatz beträgt $20 \%$. Die Rechnung wird sofort per Banküberweisung beglichen. 
\item Berechnen Sie die Abschreibungsbeträge für die Maschine aus Geschäftsvorfall 1. und buchen Sie die Abschreibung (jeweils direkt und indirekte) für das Jahr J3 im Falle einer  
\begin{itemize}
\item[a)] linearen Abschreibung auf den Restwert in Höhe von null und einer geschätzten Nutzungsdauer von 4 Jahren,  
\item[b)] Leistungsabschreibung bei einem geschätzten Leistungsvorrat von 600.000 Einheiten und einer Produktion im Jahr 3 von 75.000 Einheiten,  
\item[c)] geometrisch-degressiven Abschreibung auf den Restbuchwert von 10.000 € und einer geschätzten Nutzungsdauer von 4 Jahren,  
\item[d)] arithmetisch-degressiven Abschreibung auf den Restwert in Höhe von null und einer geschätzten Nutzungsdauer von 4 Jahren. 
\end{itemize}
\item Am 01.01.J1 wird eine Maschine zum Preis von 238.000 € (inklusive $19 \%$ Umsatzsteuer) erworben und sofort per Banküberweisung bezahlt. Diese Maschine besitzt eine geschätzte Nutzungsdauer von 10 Jahren und soll linear auf einen Restwert in Höhe von null abgeschrieben werden. Bilden Sie alle relevanten Buchungssätze in den Jahren J1, J2 und J3. Unterscheiden Sie dabei nach 
\begin{itemize}
\item[a)] direkter Abschreibung 
\item[b)] indirekter Abschreibung. 
\end{itemize} 
Wie hoch sind die Bestände auf den angesprochenen Bestandskonten in den Fällen a) und b) am Ende von Jahr 3? Gehen Sie davon aus, dass die gekaufte Maschine die einzige Maschine in dem betrachteten Unternehmen ist. Die Bestandskonten Eigenkapital, Bank und Vorsteuer sind zu vernachlässigen. 
\item Die Maschine aus Geschäftsvorfall 3. wird am 31.12.J3 verkauft (die Umsatzsteuer beträgt $19  \%$). Unterscheiden Sie hinsichtlich des Netto-Verkaufserlöses drei Fälle:
\begin{itemize}
\item[I.] 140.000 €   
\item[II.] 100.000 €   
\item[III.] 160.000 €.
\end{itemize} Wie lauten die Buchungssätze in den Fällen I. bis III., wenn die Maschine 
\begin{itemize}
\item[a)] direkt
\item[b)] indirekt
\end{itemize} 
abgeschrieben wurde? 
\item Eine Maschine, die bereits indirekt voll abgeschrieben war (kein Erinnerungswert), wird gestohlen. Die Anschaffungskosten betrugen ursprünglich 7.500 €.   
\item Am 18.09.J1 wird ein LKW für 214.200 € (inklusive $19 \%$ Umsatzsteuer) gekauft. Der Kaufpreis wird sofort per Banküberweisung beglichen. Der LKW wird linear über eine geschätzte Nutzungsdauer von 6 Jahren direkt abgeschrieben. Es ist davon auszugehen dass der Restwert des LKW am Ende der Nutzungsdauer eine Höhe von null € aufweist. Das Unternehmen nutzt die Möglichkeit 
\begin{itemize}
\item[a)] monatsgenau 
\item[b)] halbjahresgenau abzuschreiben.
\end{itemize}  
Am 15.01.J4 wird der LKW für 100.000 € netto verkauft. Bilden Sie alle relevanten Buchungssätze für die Jahre J1 bis J4.
\end{enumerate}

\underline{Aufgabe 1} \\
Technische Anlagen und Maschinen 550.000 an Bank 660.000 \\
Ust 110.000 \\

\underline{Aufgabe 2} \\
Abschreibungen: \begin{itemize}
\item periodengerechte Erfolgsermittlung
\item Aufteilung der Anschaffungskosten über die Nutzungsdauer
\end{itemize}
planmäßige Abschreibung: verbrauchsbedingt \\
außerplanmäßige Abschreibung: wirtschaftlich bedingt \\

a) lineare Abschreibung \\
$Abt = \dfrac{AK-RBW}{n} = \dfrac{550.000}{4} = 137.500$ \\
direkt: Abschreibung auf Sachanlagen an TA und M 137.500 \\
indirekt: Abschreibung auf SA an Wertberichtigungen zu SA 137.500 \\
b) \\
$Abt= \dfrac{AK}{Leistungsvorrat} *$  Leistungsabgabe in t \\
J3: $\dfrac{550.000}{600.000} * 75.000 = 68.750$ \\
direkt: \\
indirekt: \\
c) geometrisch-degressiv \\
\underline{Aufgabe 3} \\
\underline{Aufgabe 4} \\
\underline{Aufgabe 5} \\
\underline{Aufgabe 6} \\

%------------------------------------------------------
\subsection{Außerplanmäßige Abschreibungen und Zuschreibungen}
%-------------------------------------------
\paragraph{Aufgabe 2}
Bilden Sie die Buchungssätze für nachfolgende Geschäftsvorfälle: \\
\begin{enumerate}
\item Am 01.04.J1 wird von der TBR KG eine Tunnelbohrmaschine zum Preis von 550.000 € (inklusive $10 \%$ Umsatzsteuer) zur Nutzung im Unternehmen erworben und ab diesem Tag eingesetzt. Der Rechnungsbetrag wird sofort per Banküberweisung beglichen.  Zum Bilanzstichtag des folgenden Jahres am 31.12.J2 stellt sich heraus, dass bei der Tunnelbohrmaschine ein Schaden und damit ein außerplanmäßiger Abschreibungsbedarf in Höhe von 82.500 € besteht, weil von einer dauernden Wertminderung ausgegangen wird.  Im Jahr J3 wird die Tunnelbohrmaschine weiterhin genutzt. Zum 31.12.J3 kann der Schaden wider Erwarten doch behoben werden und die Gründe für die Ende des Jahres J2 durchgeführte außerplanmäßige Abschreibung entfallen. Folglich ist am 31.12.J3 eine Wertaufholung vorzunehmen.  Bilden Sie die relevanten Buchungssätze für die Jahre J1 bis J3. Die Abschreibung erfolgt linear und monatsgenau bei einer geschätzten Nutzungsdauer von zehn Jahren. Abschreibungen auf das abnutzbare Sachanlagevermögen erfolgen nach der direkten Methode. Als Maßstab für den beizulegenden Wert verwendet die TBR KG den aktuellen Marktwert der Tunnelbohrmaschine. Gehen Sie bei der Wertaufholung gemäß § 253 Abs. 5 S. 1 HGB von einem Marktwert der Tunnelbohrmaschine zum 31.12.J3 von 
\begin{itemize}
\item[a)] 370.000 € 
\item[b)] 350.000 € aus. 
\end{itemize} 
\item Am 31.12.J1 liegen 20.000 Rollen Kupferdraht auf Lager, die zu 1.000 € je Rolle beschafft wurden. Der beizulegende Wert bemisst sich nach den aktuellen Wiederbeschaffungskosten der Kupferdrahtrollen. Zum Jahresende betragen die Wiederbeschaffungskosten nur noch 850 € je Rolle. 
\item Im Rahmen der Inventur wird festgestellt, dass Waren im Wert von 20.000 € gestohlen wurden.
\end{enumerate}
%-------------------------------------------
\paragraph{Aufgabe 3}
Bilden Sie folgende Geschäftsvorfälle durch Buchungssätze ab: \\
\begin{enumerate}
\item Unser Kunde Zahlnix hat wiederholt seine Zahlungsfrist verstreichen lassen. Aus einer Pressemeldung erfahren wir, dass Zahlnix die Eröffnung des Insolvenzverfahrens beantragen möchte. Unsere Forderung gegenüber Zahlnix aus Warenlieferungen beträgt 154.700 € (inklusive $19 \%$ Umsatzsteuer). 
\item Wir erfahren weiter aus der Presse, dass für uns nur mit einer Quote von $25 \%$ unserer Forderungen gegen Zahlnix zu rechnen ist. Wir schreiben direkt ab. 
\item Nachdem unsere Forderungen gegen unseren Kunden Leerkass zweifelhaft geworden sind, haben wir diese in Höhe von 119.000 € (inklusive $19 \%$ Umsatzsteuer) bereits auf das Konto „Zweifelhafte Forderungen“ umgebucht. Nun erfahren wir, dass diese Forderungen endgültig uneinbringlich werden und schreiben sie daher direkt zu $100 \%$ ab.
\item Nach Abschluss des Insolvenzverfahrens erhalten wir eine Banküberweisung von Zahlnix. Unterscheiden Sie drei Fälle, bei denen Zahlnix 
\begin{itemize}
\item[a)] 38.675 €
\item[b)] 41.650 €
\item[c)] 29.750 € überweist.
\end{itemize} 
\item Zu unserer Überraschung überweist auch Leerkass noch 23.800 €. 
\item 
\begin{itemize}
\item[a)] Nach Einzelwertberichtigungen verbleit auf dem Konto „Forderungen aus L\&L“ zum 31.12.J1 ein Bestand in Höhe von 178.500 € im Soll (generell liegt diesen Forderungen ein Umsatzsteuersatz in Höhe von $19 \%$ zugrunde). Wir führen erstmalig eine Pauschalwertberichtigung (indirekte Buchung) mit einem unserer Meinung nach angemessenen Satz in Höhe von $3 \%$ durch.
\item[b)] Zum 31.12.J2 beträgt der Forderungsbestand nach Einzelwertberichtigungen auf dem Konto „Forderungen aus L\&L“ 261.800 €. Darin enthalten sind 95.200 € Forderungen gegenüber der öffentlichen Hand (generell liegt allen Forderungen wieder ein Umsatzsteuersatz in Höhe von $19 \%$ zugrunde). Passen Sie die Pauschalwertberichtigung an. Verwenden Sie einen Satz von $3 \%$.
\item[c)] Zum 31.12.J3 beträgt der Forderungsbestand nach Einzelwertberichtigungen auf dem Konto „Forderungen aus L\&L“ 190.400 €. Die Forderungen gegenüber der öffentlichen Hand sind mittlerweile beglichen (generell liegt allen Forderungen erneut ein Umsatzsteuersatz in Höhe von $19 \%$ zugrunde). Passen Sie die Pauschalwertberichtigung nochmals an. Verwenden Sie ebenfalls einen Satz von $3 \%$. 
\end{itemize}  
\end{enumerate}
%------------------------------------------------------
\subsection{Zeitliche Abgrenzungen}
%-------------------------------------------
\paragraph{Aufgabe 4}
Bilden Sie die Buchungssätze zu folgenden Geschäftsvorfällen. Buchen Sie jeweils am Tag des Vorfalls, am Bilanzstichtag sowie im neuen Jahr. Die Umsatzsteuer ist zu vernachlässigen. \\
\begin{enumerate}
\item Am 01.12. überweisen wir 1.500 € Miete. Davon entfallen 500 € auf das nächste Jahr. 
\item Am 29.12. erhalten wir eine vorschüssige Zinszahlung eines Schuldners per Banküberweisung. Er zahlt die Zinsen für das erste Quartal des Folgejahres. Der noch nicht getilgte Kreditbetrag liegt bei 50.000 €. Der Zinssatz beträgt $4 \%$ p.a. 
\item Für einen Bankkredit zahlen wir die Zinsen alle sechs Monate nachschüssig. Der nächste Zinszahlungstermin ist der 28. Februar des Folgejahres. Es werden 600 € zu überweisen sein. 
\item Die Miete für von uns vermietete Räume für Dezember wird von unserem Mieter erst im Januar beglichen. Er muss einen Betrag in Höhe von 5.000 € zahlen. 
\item Wir haben eine Spezialmaschine an einen Zulieferbetrieb vermietet. Die monatliche Rate beträgt 400 €. Die Miete für Dezember wird erst im Januar überwiesen. 
\item Unsere Handelsvertreter haben Anspruch auf Provision. Diese wird monatlich nachschüssig überwiesen. Die Provisionen für Dezember belaufen sich auf 20.000 €. 
\end{enumerate}
%-------------------------------------------
\paragraph{Aufgabe 5}
Am 01.01.2016 nimmt die Fanter Flash AG ein Darlehen für den Bau einer Orangen-Entsaftungsfabrik bei der C\&C Bank GmbH zu folgenden Konditionen auf:  \\
\begin{tabular}{cc}
Nominalbetrag: & 1.000.000 € \\
Auszahlungskurs & $96,4 \%$ \\
Laufzeit: & 8 Jahre \\
Nominalzins: & $4 \%$ p.a. \\
Zinszahlung: & jährlich nachschüssig, beginnend mit dem 01.01.2017
\end{tabular} \\
Die C\&C Bank GmbH überweist den Auszahlungsbetrag auf das Bankkonto der Fanter Flash AG. Die fälligen Zins- und ggf. Tilgungsleistungen für das vorangegangene Jahr werden jeweils am 01.01. per Banküberweisung geleistet. Das Disagio soll unter der Zielsetzung eines möglichst hohen Jahresüberschusses im Jahr 2016 behandelt werden.  \\
Bilden Sie sämtliche Buchungssätze (ausgenommen die formalen Abschlussbuchungen der angesprochenen Konten), die im Zusammenhang mit dem Darlehen bei der Fanter Flash AG in den Jahren 2016 und 2017 anfallen. Gehen Sie davon aus, dass die Tilgungszahlungen am Ende der Kreditlaufzeit auf einmal vollständig geleistet werden (endfälliger Kredit). Wie müsste das Disagio behandelt werden, wenn 2016 ein möglichst niedriger Jahresüberschuss angestrebt wird? \\
%-------------------------------------------
\paragraph{Aufgabe 5}
Am 01.01.2016 nimmt die Fanter Flash AG ein Darlehen für den Bau einer Orangen-Entsaftungsfabrik bei der C\&C Bank GmbH zu folgenden Konditionen auf:  \\
\begin{tabular}{cc}
Nominalbetrag: & 1.000.000 € \\
Auszahlungskurs & $96,4 \%$ \\
Laufzeit: & 8 Jahre \\
Nominalzins: & $4 \%$ p.a. \\
Zinszahlung: & jährlich nachschüssig, beginnend mit dem 01.01.2017
\end{tabular} \\
Die C\&C Bank GmbH überweist den Auszahlungsbetrag auf das Bankkonto der Fanter Flash AG. Die fälligen Zins- und ggf. Tilgungsleistungen für das vorangegangene Jahr werden jeweils am 01.01. per Banküberweisung geleistet. Das Disagio soll unter der Zielsetzung eines möglichst hohen Jahresüberschusses im Jahr 2016 behandelt werden.  \\
Bilden Sie sämtliche Buchungssätze (ausgenommen die formalen Abschlussbuchungen der angesprochenen Konten), die im Zusammenhang mit dem Darlehen bei der Fanter Flash AG in den Jahren 2016 und 2017 anfallen. Gehen Sie davon aus, dass die Tilgungszahlungen am Ende der Kreditlaufzeit auf einmal vollständig geleistet werden (endfälliger Kredit). Wie müsste das Disagio behandelt werden, wenn 2016 ein möglichst niedriger Jahresüberschuss angestrebt wird? \\
%-------------------------------------------
\paragraph{Aufgabe 6}
Buchen Sie die folgenden Geschäftsvorfälle. Wie lauten die Buchungssätze zum Bilanzstichtag und im neuen Geschäftsjahr? \\
\begin{enumerate}
\item Aufgrund eines anhängigen Gerichtsverfahrens müssen wir zum 31.12. damit rechnen, zur Zahlung von Gerichts- und Anwaltskosten in Höhe von 30.000 € herangezogen zu werden.   Das Gerichtsverfahren wird im März des neuen Jahres abgeschlossen. Wir erhalten 
\begin{itemize}
\item[a)] eine Rechnung über 20.000 € Gerichtskosten sowie 15.000 € Anwaltsgebühren zuzüglich 2.850 € Umsatzsteuer, die sofort per Banküberweisung beglichen wird. 
\item[b)] eine Nachricht, dass unser Gegner die Kosten zu tragen hat. 
\item[c)] eine Nachricht, dass wir binnen 60 Tagen 50.000 € Schadenersatz (ohne Umsatzsteuer) zu zahlen haben. Diesen Betrag überweisen wir am letzten Tag der 60-TageFrist. 
\end{itemize}
\item Zum Bilanzstichtag wird für das laufende Jahr mit einer Gewerbesteuernachzahlung in Höhe von 7.500 € gerechnet. Der Steuerbescheid geht uns im nächsten Jahr zu und lautet über einen Nachzahlungsbetrag von
\begin{itemize}
\item[a)] 6.000 €, 
\item[b)] 7.500 €, 
\item[c)] 8.000 €,  
\end{itemize} 
der jeweils sofort per Banküberweisung beglichen wird. 
\item Für die von uns verkauften Produkte im Wert von 300.000 € (netto) ist die von uns (freiwillig) eingeräumte Garantiefrist noch nicht abgelaufen. Wir gehen davon aus, dass Garantieleistungen in Höhe von $1 \%$ des (Netto-) Verkaufswerts erbracht werden müssen. Bis zum Ablauf der Garantiefrist Ende März des folgenden Jahres sind tatsächlich Garantiefälle im Wert von 3.500 € eingetreten. Wir überweisen diesen Betrag zuzüglich $19 \%$ Umsatzsteuer an unsere Vertragswerkstätten, die die Reparaturleistungen ausführen. 
\item Im abgelaufenen Geschäftsjahr ist eine dringend notwendige Gebäudesanierung unterlassen worden. Die Arbeiten sollen sofort zu Beginn des neuen Geschäftsjahres nachgeholt werden. Die Bauunternehmung macht einen Kostenvoranschlag in Höhe von 450.000 € zuzüglich $19 \%$ Umsatzsteuer. Nach Abschluss der Sanierungsarbeiten überweisen wir den Rechnungsbetrag in Höhe von 416.500 € (brutto). 
\end{enumerate}
%-------------------------------------------
\paragraph{Aufgabe 7}
Im Jahr 2015 weichen die handelsrechtlichen und steuerlichen Wertansätze der Warhola AG für einen selbst geschaffenen immateriellen Vermögensgegenstand des Anlagevermögens in Form eines Bildbearbeitungsprogramms voneinander ab. Bei der Warhola AG handelt es sich um eine große Kapitalgesellschaft. Die Warhola AG nutzt das Wahlrecht nach § 248 Abs. 2 HGB und aktiviert zum 01.01.2015 in der Handelsbilanz die Herstellungskosten für das Bildbearbeitungsprogramm in Höhe von 240.000 €. Das Programm wird in der Handelsbilanz über drei Jahre (2015 bis 2017) linear abgeschrieben. In der Steuerbilanz werden im Gegenzug dazu 2015 die Kosten vollständig als Aufwendungen erfasst.  Bis auf die abweichenden Ansätze für das Bildbearbeitungsprogramm stimmen die Wertansätze für das restliche Vermögen und Schulden in Handels- und Steuerbilanz in den drei Jahren vollständig überein.
Der in der Handelsbilanz ausgewiesene Gewinn vor Steuern beträgt im Jahr 2015 800.000 €. Im Jahr 2016 beläuft sich der Steuerbilanzgewinn vor Berücksichtigung von Steuern auf 540.000 €. Im Jahr 2017 gleichen sich die unterschiedlichen Wertansätze wieder aus, weil das Programm vollständig abgeschrieben wird: Handelsbilanzgewinn vor Steuern 300.000 €, Steuerbilanzgewinn vor Berücksichtigung von Steuern 380.000 €. Der Steuersatz der Warhola AG beträgt in allen Jahren $30 \%$. Das Ansatzwahlrecht zur Bildung aktiver latenter Steuern nach § 274 Abs. 1 S. 2 HGB wird von der Warhola AG dahingehend ausgeübt, dass aktive latente Steuern in der Handelsbilanz stets angesetzt werden, wenn dies gesetzlich zulässig ist. Die Steuerzahlungen werden jeweils bereits am Ende der drei Jahre vom betrieblichen Bankkonto überwiesen. 
\begin{itemize}
\item[a)] Stellen Sie den Bilanzansatz des Bildbearbeitungsprogramms jeweils zum 01.01.2015 und zum 31.12. der Jahre 2015 bis 2017 dar. 
\item[b)] Wie hoch sind jeweils der Handelsbilanzgewinn vor Steuern und der Steuerbilanzgewinn vor Berücksichtigung von Steuern in den Jahren 2015 bis 2017? Wie hoch sind die mit dem Bildbearbeitungsprogramm verbundenen Abschreibungen in der Handels- und Aufwendungen in der Steuerbilanz jeweils in den Jahren 2015 bis 2017? 
\item[c)] Sind Ende 2015 aktive oder passive latente Steuern in der Handelsbilanz zu berücksichtigen? Begründen Sie Ihre Antwort kurz, indem Sie auf die Kriterien zum Ansatz latenter Steuern eingehen. Gilt ihre Antwort auch für das Jahr 2016
\item[d)] Wie werden die Steuern (einschließlich der latenten Steuern) Ende 2015, 2016 und Ende 2017 gebucht? 
\end{itemize}
%----------------------------------------------------------------------------------------
\section{Jahresabschluss}
%-------------------------------------------
\paragraph{Aufgabe 1}
An der eBought OHG sind die Gesellschafter Kermit Samber, Reinhild Andersen und Peter Elliot beteiligt. Der Eröffnungsbilanz des Geschäftsjahres 2017 ist zu entnehmen, dass der Bestand auf dem Kapitalkonto von Herrn Samber 120.000 €, der von Frau Andersen 80.000 € und der von Herrn Elliot 100.000 € beträgt. Herr Samber entnimmt nach sechs Monaten (30.06.2017) mit Zustimmung der beiden anderen Gesellschafter einen Betrag in Höhe von 40.000 €. Frau Andersen legt im Gegensatz dazu nach drei Monaten (01.04.2017) einen Betrag in Höhe von 20.000 € ein.  Der Gesellschaftsvertrag der eBought OHG sieht in Bezug auf die Gewinnverwendung vor, dass Herr Samber stets eine Vorabvergütung in Höhe von 5.000 € für seinen höheren Arbeitseinsatz in der eBought OHG erhält. Weiterhin werden gemäß des Gesellschaftsvertrags nach erfolgter Vorabvergütung von Herrn Samber die Kapitalanteile mit $4 \%$ verzinst. Sofern der Jahresüberschuss dazu nicht ausreicht, kommt ein entsprechend niedrigerer Satz zur Anwendung. Einlagen und Entnahmen werden im Jahr der Einlage bzw. Entnahme bei der Gewinnverteilung zeitanteilig berücksichtigt. Der restliche Gewinn wird nach Köpfen verteilt.  Der Gewinn im Geschäftsjahr 2017 beträgt 91.800 €. \\
\begin{itemize}
\item[a)] Ermitteln Sie die Schlussbestände der jeweiligen Eigenkapitalkonten zum 31.12.2017, indem Sie die Tabelle aus der Lösungshilfe gemäß den Gewinnverwendungsregeln des Gesellschaftsvertrags vollständig ausfüllen.
\item[b)] Ermitteln Sie die Buchungssätze für die Gewinnverteilung des Jahres 2017. Schreiben Sie die anteiligen Gewinne zunächst den Privatkonten der Gesellschafter gut. Die Einlagen und Entnahmen sind vor der Buchung der Gewinnverteilung bereits buchhalterisch auf den Privatkonten berücksichtigt worden. Danach ist der Abschluss der Privat- und anschließend der Eigenkapitalkonten vorzunehmen. 
\end{itemize}

a) Gewinnverteilungsübersicht \\
\begin{tabular}{|c|c|c|c|c|c|c|c|c|}
\hline
Gesell- & AK & Kapital- & Sonder- & Restgewinn- & Gesamt- & Ein- & Kapitalver- & EK \\
schafter &  & verzinsung & vergütung & anteil & gewinn & lagen & änderung & \\\hline
Samber & 120.000 & 4.000 & 5.000 & 25.000 & 34.000 & -40.000 & -6.000 & 114.000 \\\hline
Anderesen & 80.000 & 3.800 & - & 25.000 & 28.800 & +20.000 & 48.800 & 128.800 \\\hline
Elliot & 100.000 & 4.000 & - & 25.000 & 29.000 & - & 29.000 & 129.000 \\\hline
Summe & 300.000 & 11.800 & 5.000 & 75.000 & 91.800 & -20.000 & 71.800 & 371.800 \\\hline
\end{tabular}

Kapitalverzinsung $4 \%$ \\
Samber: $(120.000-40.000 * \dfrac{6}{12}) *0.04 = 4.000$ \\
Andersen: $(80.000+20.000 * \dfrac{9}{12}) * 0.04 = 3.800$ \\
Elliot: $100.000 * 0.04 = 4.000$ \\

nach Köpfen verteilt: $91.800 - 11.800 - 5.000 = 75.000$ \\
$\rightarrow$ jeder bekommt 25.000
%-------------------------------------------
\paragraph{Aufgabe 2}
\begin{itemize}
\item[a)] Eine Aktiengesellschaft hat im Jahr J1 einen Jahresüberschuss in Höhe von 120.000 € erwirtschaftet. Nehmen Sie die entsprechenden Abschlussbuchungen vor. 
\item[b)] Im Jahr J2 wird beschlossen, $75 \%$ des Gewinns des Vorjahres zu thesaurieren und den Rest an die Aktionäre auszuschütten. Buchen Sie die Gewinnverwendung. 
\end{itemize}
%----------------------------------------------------------------------------------------
\section{Sonstige}
\textbf{Soll und Haben}
\begin{tabular}{cc|cc}
S & & & H \\\hline
  & & & \\
  & & & \\\hline
  & & & \\\hline
\end{tabular}

\textbf{Akitva und Passiva}
\begin{tabular}{cc|cc}
A & & & P \\\hline
  & & & \\
  & & & \\\hline
  & & & \\\hline
\end{tabular}

\subsection{Abkürzungen}
BV Bestandsveränderung \\
ZM Zahlungsmittel \\
GV Geldvermögen \\
RV Reinvermögen \\
AT Aktivtausch \\
PT Passivtausch \\
AP+ Aktiv-Passiv-Mehrung \\
AP- Aktiv-Passiv-Minderung \\
EBK Eröffnungsbilanzkonto \\
SBK Schlussbilanzkonto \\
Grund+Bauten Grundstück+Bauten \\
BGA Betriebs+GA/ Betriebs+Geschäftsausstattung \\
FLL Forderung aus LuL \\
EK-Konto Eigenkapitalkonto \\
VLL Verbindlichkeiten aus LuL \\
VKI Verbindlichkeiten gegenüber KI \\

\subsection{Rechnungsschritte}
\subsubsection{Ust-Berechnung}
Bsp. Ust=$10 \%$ \\
Brutto $\rightarrow$ Netto:  /1,1 \\
Netto $\rightarrow$ Brutto:  *1,1 \\
Netto $\rightarrow$ Ust/Vst:  *0,1 \\
Brutto $\rightarrow$ Ust/Vst:  /1,1*0,1 \\

\subsubsection{Skonto-Berechnung}
Bsp. Skonto $2 \%$ \\
Skontoaufwand/-ertrag: Netto * Skonto, also Netto * 0,02 \\
Ust/Vst Korrektur: Ust*Skonto, also Ust* 0.02 \\

\subsubsection{Kunde überweist einen Betrag}
\begin{enumerate}
\item unter $2 \%$ Skontoabzug \\
Betrag/0,98 $\rightarrow$ Brutto \\
Betrag/0,98/1,1 $\rightarrow$ Netto \\
Skontoaufwand/-ertrag: Betrag/0,98/1,1*0.02
\item erhält $20 \%$ Rabatt und Ust=$10 \%$\\
Betrag/0,8 $\rightarrow$ Brutto \\
Betrag/0,8/1,1 $\rightarrow$ Netto \\
\end{enumerate}
%----------------------------------------------------------------------------------------
\end{document}