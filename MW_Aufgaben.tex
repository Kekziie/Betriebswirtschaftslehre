\documentclass[paper=a4, fontsize=11pt]{scrartcl}
\usepackage[utf8]{inputenc}
\usepackage{amsmath}
\usepackage{amsfonts}
\usepackage{amssymb}
\author{Kim Thuong Ngo}


\usepackage[T1]{fontenc}
\usepackage{fourier}

\usepackage{lipsum}

\usepackage{listings}
\usepackage{graphicx}
\usepackage{tabularx}

\usepackage{sectsty}
\allsectionsfont{\centering \normalfont\scshape}

\usepackage{fancyhdr}
\pagestyle{fancyplain}
\fancyhead{}
\fancyfoot[L]{}
\fancyfoot[C]{}
\fancyfoot[R]{\thepage}
\renewcommand{\headrulewidth}{0pt}
\renewcommand{\footrulewidth}{0pt}
\setlength{\headheight}{13.6pt}

\numberwithin{equation}{section}
\numberwithin{figure}{section}
\numberwithin{table}{section}

\setlength\parindent{0pt}

\newcommand{\horrule}[1]{\rule{\linewidth}{#1}}

\title{
\normalfont \normalsize
\textsc{Mathematische Methoden der Wirtschaftswissenschaften} \\ [25pt]
\horrule{0.5pt} \\[0.4cm]
\huge Aufgaben \\
\horrule{2pt} \\[0.5cm]
}

\author{Kim Thuong Ngo}

\date{\normalsize\today}

\begin{document}

\maketitle

\newpage

\tableofcontents

%----------------------------------------------------------------------------------------
\newpage
\section{Tutorium I}
%----------------------------------------------------
\subsection{Summennotation}
Berechnen Sie folgende Summen:
%-----------------------------------
\paragraph{a)}
$$\sum^{15}_{j=12}j$$

%-----------------------------------
\paragraph{b)}
$$\sum^{6}_{i=1}5i$$

%-----------------------------------
\paragraph{c)}
$$\sum^{10}_{i=1}(i^{2}-i)-\sum^{9}_{i=1}i^{2}+\sum^{10}_{k=2}k$$

%-----------------------------------
\paragraph{d)}
$$\sum^{5}_{i=0}(e^{\pi *i} \sqrt{i+1})-\sum^{10}_{j=6}(e^{\pi *(j-5)} \sqrt{j-4})$$

%-----------------------------------
\paragraph{e)}
$$\sum^{3}_{k=-2}k*i+1^{k}$$

%-----------------------------------
\paragraph{f)}
$$\sum^{3}_{i=1} \sum^{3}_{j=0}i*2^{j}$$

%----------------------------------------------------
\subsection{Produktnotation}
Berechnen Sie die folgenden Produkte:
%-----------------------------------
\paragraph{a)}
$$\Pi^{3}_{m=1}*(-1)^{m}$$

%-----------------------------------
\paragraph{b)}
$$\Pi^{4}_{i=2} \Pi^{3}_{j=2} \dfrac{i}{j}$$

%----------------------------------------------------
\subsection{Binomialkoeffizienten}
Bestimmen Sie:
%-----------------------------------
\paragraph{a)}
$$(a+b)^{7}$$

%-----------------------------------
\paragraph{b)}
$$(x-y)^{9}$$

%----------------------------------------------------
\subsection{Binomialkoeffizienten}
Bestimmen Sie
%-----------------------------------
\paragraph{a)}
$$\left(\begin{array}{c} 10 \\ 4 \end{array}\right)$$

%-----------------------------------
\paragraph{b)}

%-----------------------------------
\paragraph{c)}

%----------------------------------------------------
\subsection{Summennotation/Binomialkoeffizienten}

%----------------------------------------------------
\subsection{Folgen und Reihen}

%----------------------------------------------------
\subsection{Summennotation/ Folgen und Reihen}

%----------------------------------------------------------------------------------------
\end{document}