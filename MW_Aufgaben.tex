\documentclass[paper=a4, fontsize=11pt]{scrartcl}
\usepackage[utf8]{inputenc}
\usepackage{amsmath}
\usepackage{amsfonts}
\usepackage{amssymb}
\author{Kim Thuong Ngo}


\usepackage[T1]{fontenc}
\usepackage{fourier}

\usepackage{lipsum}

\usepackage{listings}
\usepackage{graphicx}
\usepackage{tabularx}

\usepackage{sectsty}
\allsectionsfont{\centering \normalfont\scshape}

\usepackage{fancyhdr}
\pagestyle{fancyplain}
\fancyhead{}
\fancyfoot[L]{}
\fancyfoot[C]{}
\fancyfoot[R]{\thepage}
\renewcommand{\headrulewidth}{0pt}
\renewcommand{\footrulewidth}{0pt}
\setlength{\headheight}{13.6pt}

\numberwithin{equation}{section}
\numberwithin{figure}{section}
\numberwithin{table}{section}

\setlength\parindent{0pt}

\newcommand{\horrule}[1]{\rule{\linewidth}{#1}}

\title{
\normalfont \normalsize
\textsc{Mathematische Methoden der Wirtschaftswissenschaften} \\ [25pt]
\horrule{0.5pt} \\[0.4cm]
\huge Aufgaben \\
\horrule{2pt} \\[0.5cm]
}

\author{Kim Thuong Ngo}

\date{\normalsize\today}

\begin{document}

\maketitle

\newpage

\tableofcontents

%----------------------------------------------------------------------------------------
\newpage
\section{Tutorium I}
%----------------------------------------------------
\subsection{Summennotation}
Berechnen Sie folgende Summen:
%-----------------------------------
\paragraph{a)}
$$\sum^{15}_{j=12}j$$

$\sum^{15}_{j=12}j = 12+13+14+15 = 54$

%-----------------------------------
\paragraph{b)}
$$\sum^{6}_{i=1}5i$$

$\sum^{6}_{i=1}5i = 5*1+5*2+5*3+5*4+5*5+5*6 = 5+10+15+20+25+30 = 105$

%-----------------------------------
\paragraph{c)}
$$\sum^{10}_{i=1}(i^{2}-i)-\sum^{9}_{i=1}i^{2}+\sum^{10}_{k=2}k$$

$\sum^{10}_{i=1}(i^{2}-i)-\sum^{9}_{i=1}i^{2}+\sum^{10}_{k=2}k $ \\
$= (1^{2}-1)+(2^{2}-2)+(3^{2}-3)+(4^{2}-4)+(5^{2}-5)
+(6^{2}-6)+(7^{2}-7)+(8^{2}-8)+(9^{2}-9)
+(10^{2}-10)
-(1^{2}+2^{2}+3^{2}+4^{2}+5^{2}+6^{2}
+7^{2}+8^{2}+9^{2})
+(2+3+4+5+6+7+8+9+10)$ \\
$=
(0+2+6+12+20+30+42+56+72+90)
-(1+4+9+16+25+36+49+64+81)
+(2+3+4+5+6+7+8+9+10)$ \\
$=
330-285+54 = 99$

%-----------------------------------
\paragraph{d)}
$$\sum^{5}_{i=0}(e^{\pi *i} \sqrt{i+1})-\sum^{10}_{j=6}(e^{\pi *(j-5)} \sqrt{j-4})$$

$\sum^{5}_{i=0}(e^{\pi *i} \sqrt{i+1})-\sum^{10}_{j=6}(e^{\pi *(j-5)} \sqrt{j-4})$ \\
$=
e^{\pi *0} \sqrt{0+1} + e^{\pi *1} \sqrt{1+1}
+ e^{\pi *2} \sqrt{2+1} + e^{\pi *3} \sqrt{3+1}
+ e^{\pi *4} \sqrt{4+1} + e^{\pi *5} \sqrt{5+1} 
- (e^{\pi *(6-5)} \sqrt{6-4} + e^{\pi *(7-5)} \sqrt{7-4} + e^{\pi *(8-5)} \sqrt{8-4} 
+ e^{\pi *(9-5)} \sqrt{9-4} + e^{\pi *(10-5)} \sqrt{10-4})$ \\
$=
e^{0} \sqrt{1} + e^{\pi} \sqrt{2} +  e^{2* \pi} \sqrt{3} +  e^{3* \pi} \sqrt{4} +  e^{4* \pi} \sqrt{5}
+  e^{5* \pi} \sqrt{6}
-( e^{\pi} \sqrt{2} +  e^{2* \pi} \sqrt{3}
+  e^{3* \pi} \sqrt{4} +  e^{4* \pi} \sqrt{5}
+  e^{5* \pi} \sqrt{6})  $ \\
$=
e^{0} \sqrt{1} = 1*1 = 1 $

%-----------------------------------
\paragraph{e)}
$$\sum^{3}_{k=-2}k*i+1^{k}$$

$\sum^{3}_{k=-2}k*i+1^{k}$ \\
$=
(-2)*i+1^{-2} + (-1)*i+1^{-1} + (0)*i+1^{0}
+ 1*i+1^{1} + 2*i+1^{2} + 3*i+1^{3}$ \\
$=
(-2i+1)+(-i+1)+1+(i+1)+(2i+1)+(3i+1) = 3i+6
$
%-----------------------------------
\paragraph{f)}
$$\sum^{3}_{i=1} \sum^{3}_{j=0}i*2^{j}$$

$\sum^{3}_{i=1} \sum^{3}_{j=0}i*2^{j}$ \\
$=
(1*2^{0}+2*2^{0}+3*2^{0})
+(1*2^{1}+2*2^{1}+3*2^{1})
+(1*2^{2}+2*2^{2}+3*2^{2})
+(1*2^{3}+2*2^{3}+3*2^{3})$ \\
$=
(1+2+3)+(2+4+6)+(4+8+12)+(8+16+24) = 90$
%----------------------------------------------------
\subsection{Produktnotation}
Berechnen Sie die folgenden Produkte:
%-----------------------------------
\paragraph{a)}
$$\Pi^{3}_{m=1}*(-1)^{m}$$

$\Pi^{3}_{m=1}*(-1)^{m}$ \\
$=1*(-1)^{1} *2*(-1)^{2} * 3*(-1)^{3}$ \\
$= -1* 2 *-3 = 6$
%-----------------------------------
\paragraph{b)}
$$\Pi^{4}_{i=2} \Pi^{3}_{j=2} \dfrac{i}{j}$$

$\Pi^{4}_{i=2} \Pi^{3}_{j=2} \dfrac{i}{j}$ \\
$= \dfrac{2}{2} * \dfrac{2}{3}
* \dfrac{3}{2} * \dfrac{3}{3} 
* \dfrac{4}{2} * \dfrac{4}{3}$ \\
$= \dfrac{2*2*3*3*4*4}{2*2*2*3*3*3}
= \dfrac{4*4}{2*3}
= \dfrac{2*4}{3}
= \dfrac{8}{3}$
%----------------------------------------------------
\subsection{Binomialkoeffizienten}
Bestimmen Sie:
%-----------------------------------
\paragraph{a)}
$$(a+b)^{7}$$

%-----------------------------------
\paragraph{b)}
$$(x-y)^{9}$$

%----------------------------------------------------
\subsection{Binomialkoeffizienten}
Bestimmen Sie
%-----------------------------------
\paragraph{a)}
$$\left(\begin{array}{c} 10 \\ 4 \end{array}\right)$$

%-----------------------------------
\paragraph{b)}
$$\left(\begin{array}{c} 201 \\ 198 \end{array}\right)$$

%-----------------------------------
\paragraph{c)}
$$\left(\begin{array}{c} 23 \\ 4 \end{array}\right)+\left(\begin{array}{c} 23 \\ 5 \end{array}\right)$$

%----------------------------------------------------
\subsection{Summennotation/Binomialkoeffizienten}
%-----------------------------------
\paragraph{a)}
$$\sum^{3}_{k=0} \left(\begin{array}{c} 3 \\ k \end{array}\right)$$

%-----------------------------------
\paragraph{b)}
$$\sum^{11}_{k=0} \left(\begin{array}{c} 11 \\ k \end{array}\right) (-2)^{11-k} 3^{k+3}$$

%----------------------------------------------------
\subsection{Folgen und Reihen}
Die Glieder einer arithmetischen Folge sind definiert als $a_{n}=a_{1}+(n-1)d$ für $n \in \mathbb{N}$ und $a_{1}=c$. Die Differenz zweier benachbarter Glieder $a_{n+1}-a_{n}$ ist konstant und gleich d. Prüfen Sie, ob es sich im folgenden um eine arithmetische Folge handelt. Wenn ja, bestimmen sie d und c. Wenn nein, so versuchen Sie eine alternative Bestimmungsgleichung anzugeben.
%-----------------------------------
\paragraph{a)}
$$2,4,6,8,10,...$$

%-----------------------------------
\paragraph{b)}
$$12,0,-12,-24,-36,...$$

%-----------------------------------
\paragraph{c)}
$$1,7,17,31,49,...$$

%----------------------------------------------------
\subsection{Summennotation/ Folgen und Reihen}
Die Summen der ersten n Gleider einer Zahlenfolge heißen n-te Partialsumme.
%-----------------------------------
\paragraph{a)}
Schreiben Sie n-te Partialsumme in der Summennotation auf.

%-----------------------------------
\paragraph{b)}
Leiten Sie daraus eine einfache Berechnungsformel der n-ten Partialsumme einer arithmetischen Folge her.

%-----------------------------------
\paragraph{c)}
Berechnen Sie damit die n-te und die zwanzigste Partialsumme der in Aufgabe 6 genannten Folgen.

%----------------------------------------------------------------------------------------
\end{document}