\documentclass[paper=a4, fontsize=11pt]{scrartcl}
\usepackage[utf8]{inputenc}
\usepackage{amsmath}
\usepackage{amsfonts}
\usepackage{amssymb}
\author{Kim Thuong Ngo}


\usepackage[T1]{fontenc}
\usepackage{fourier}

\usepackage{lipsum}

\usepackage{listings}
\usepackage{graphicx}
\usepackage{tabularx}

\usepackage{sectsty}
\allsectionsfont{\centering \normalfont\scshape}

\usepackage{fancyhdr}
\pagestyle{fancyplain}
\fancyhead{}
\fancyfoot[L]{}
\fancyfoot[C]{}
\fancyfoot[R]{\thepage}
\renewcommand{\headrulewidth}{0pt}
\renewcommand{\footrulewidth}{0pt}
\setlength{\headheight}{13.6pt}

\numberwithin{equation}{section}
\numberwithin{figure}{section}
\numberwithin{table}{section}

\setlength\parindent{0pt}

\newcommand{\horrule}[1]{\rule{\linewidth}{#1}}

\title{
\normalfont \normalsize
\textsc{Einführung in die Betriebswirtschaftslehre} \\ [25pt]
\horrule{0.5pt} \\[0.4cm]
\huge Skript \\
\horrule{2pt} \\[0.5cm]
}

\author{Kim Thuong Ngo}

\date{\normalsize\today}

\begin{document}

\maketitle

\newpage

\tableofcontents

%----------------------------------------------------------------------------------------

\newpage

\section{Gegenstand der Betriebswirtschaftslehre}

%-------------------------------------------------------------------


%----------------------------------------------------------------------------------------

\newpage

\section{Grundzüge der Entscheidungs- und Spieltheorie}

%----------------------------------------------------------------------------------------

\newpage

\section{Warum Unternehmen?}

%----------------------------------------------------------------------------------------

\newpage

\section{Unternehmungsverfassung (Corporate Governance)}

%----------------------------------------------------------------------------------------

\newpage

\section{Literaturhinweise}

\begin{itemize}
\item Einführung in die Betriebswirtschaftslehre aus institutionenökonomischer Sicht
10. Auflage \\
Neus, Werner 
\item Ökonomik, eine Einführung
2. Auflage \\
Homann, Karl/ Suchanek, Andreas
\item Organisation und Management
6.Auflage \\
Kräkel, Matthias
\item Economics, Organization and Management
6.Auflage \\
\end{itemize}

%----------------------------------------------------------------------------------------
\end{document}
